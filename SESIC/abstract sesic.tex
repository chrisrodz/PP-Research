\documentclass{article}

%\documentclass{proc-l}

\usepackage{amssymb}
\usepackage{amsmath}
\usepackage{amsfonts}
\usepackage{amsthm}



\newcommand{\C}{\mathcal{C}}
\newcommand{\sN}{\mathcal{N}}
\newcommand{\M}{\mathcal{M}}
\newcommand{\X}{\mathbf{X}}
\newcommand{\T}{\mathcal{T}}

\newcommand{\vv} {{\boldsymbol{\nu}}}
\newcommand{\x}{{\bf x}}
\newcommand{\xx}{\mathbf{x}}
\newcommand{\be}{{\bf e}}
\newcommand{\ff}{{\mathbb{ F\!}}}
\newcommand{\lff}{{\mathbb{ F\!}}^{\,\prime}}
\newcommand{\pp}{{\mathbb{P\!}}}
\newcommand{\Fq}{{\mathbb{F\!}_q}}
\newcommand{\Fp}{{\mathbb{F\!}_p}}
\newcommand{\FF}{\mathbb{F}_2}
\newcommand{\F}{\mathbb{F}}
\newcommand{\QQ}{\mathbb{Q}_2}
\newcommand{\Qp}{\mathbb{Q}_p}
\newcommand{\Q}{\mathbb{Q}}
\newcommand{\Z}{\mathbb{Z}}
\newcommand{\Zp}{\mathbb{Z}_p}
\newcommand{\K}{K}
\newcommand{\N}{\mathbb{N}}
\newcommand{\p}{$p$\nobreakdash}
%\newcommand{\e}{\mathbf{e}}
\newcommand{\tvec}{\mathbf{t}}
\newcommand{\OKxi}{\mathcal{O}_{\K(\xi)}}
\DeclareMathOperator{\tr}{Tr}
\def\Tr{\mathop{\rm Tr}\nolimits}
\newcommand{\cc}{\mathcal{C}}
\newcommand{\il}{\mathcal{I}}
\newcommand{\ee}{\epsilon}
\newcommand{\bb}{\beta}

%%%%%%%%%%%%%%%%%%%%%%%%
% AMS Proc Styles
%%%%%%%%%%%%%%%%%%%%%%

\newtheorem{theorem}{Theorem}[section]
\newtheorem{lemma}[theorem]{Lemma}

\theoremstyle{definition}
\newtheorem{definition}[theorem]{Definition}

%\theoremstyle{corollary}
\newtheorem{corollary}[theorem]{Corollary}

\newtheorem{example}[theorem]{Example}
\newtheorem{xca}[theorem]{Exercise}

\newtheorem{construction}[theorem]{Construction}


\newtheorem{prop}[theorem]{Proposition}

\theoremstyle{remark}
\newtheorem{remark}[theorem]{Remark}

\numberwithin{equation}{section}




\begin{document}

\title{Construcci\'{o}n de Trinomios de Permutaci\'{o}n sobre Cuerpos Finitos.}
\author{Christian A. Rodr\'{\i}guez \\ Alex D. Santos \\ Universidad de Puerto Rico \\ Recinto de R\'{\i} \\ Departamento de Ciencia de C\'{o}mputos}

\maketitle

\begin{abstract}
Dado un trinomio de la forma $f_{a,b}(X)=X^r( X^{\frac{q-1}{d_1}}+aX^{\frac{q-1}{d_2}}+b )$ sobre un cuerpo finito $\mathbb{F}_q$ con tama\~{n}o de value set $s$, constru\'{\i}mos $d=lcm(d_1, d_2)$ otros trinomios en $\mathbb{F}_q$ con el mismo tama\~{n}o de value set. En particular, dado un polinomio de permutaci\'{o}n de la forma $f_{a,b}$, constru\'{\i}mos $d=lcm(d_1,d_2)$ otros polinomios de permutaci\'{o}n en $\mathbb{F}_q$. Tambi\'{e}n constru\'{\i}mos secuencias $P_{q^{m_1}}, P_{q^{m_2}}, \cdots,$ donde $P_{q^{m_i}}$ es un polinomio de permutaci\'{o}n en $\mathbb{F}_{q^{m_i}}$.
\end{abstract}

\end{document}