\documentclass[landscape,final,paperwidth=24in,paperheight=36in]{baposter}

\usepackage{calc}
\usepackage{graphicx}
\usepackage{amsmath}
\usepackage{amssymb}
\usepackage{relsize}
\usepackage{multirow}
\usepackage{rotating}
\usepackage{bm}
\usepackage{url} 

\usepackage{amsthm} 
\usepackage{amssymb}
\usepackage{amsmath}

\usepackage{graphicx}
\usepackage{spanish}{babel}

\newtheorem{corolario}{Corolario}
\newtheorem{conjetura}{Conjetura}
\newtheorem*{ejemplo*}{Ejemplo}
\newtheorem*{definicion*}{Definici\'on}
\newtheorem{definicion}{Definici\'on}
\newtheorem{proposicion}{Proposici\'on}
\newtheorem{lema}{Lema}
\newtheorem{teorema}{Teorema}

\usepackage{graphicx}
\usepackage{multicol}

%\usepackage{times}
%\usepackage{helvet}
%\usepackage{bookman}
\usepackage{palatino}

\newcommand{\captionfont}{\footnotesize}

\graphicspath{{images/}{../images/}}
\usetikzlibrary{calc}

\newcommand{\SET}[1]  {\ensuremath{\mathcal{#1}}}
\newcommand{\MAT}[1]  {\ensuremath{\boldsymbol{#1}}}
\newcommand{\VEC}[1]  {\ensuremath{\boldsymbol{#1}}}
\newcommand{\Video}{\SET{V}}
\newcommand{\video}{\VEC{f}}
\newcommand{\track}{x}
\newcommand{\Track}{\SET T}
\newcommand{\LMs}{\SET L}
\newcommand{\lm}{l}
\newcommand{\PosE}{\SET P}
\newcommand{\posE}{\VEC p}
\newcommand{\negE}{\VEC n}
\newcommand{\NegE}{\SET N}
\newcommand{\Occluded}{\SET O}
\newcommand{\occluded}{o}

%%%%%%%%%%%%%%%%%%%%%%%%%%%%%%%%%%%%%%%%%%%%%%%%%%%%%%%%%%%%%%%%%%%%%%%%%%%%%%%%
%%%% Some math symbols used in the text
%%%%%%%%%%%%%%%%%%%%%%%%%%%%%%%%%%%%%%%%%%%%%%%%%%%%%%%%%%%%%%%%%%%%%%%%%%%%%%%%

%%%%%%%%%%%%%%%%%%%%%%%%%%%%%%%%%%%%%%%%%%%%%%%%%%%%%%%%%%%%%%%%%%%%%%%%%%%%%%%%
% Multicol Settings
%%%%%%%%%%%%%%%%%%%%%%%%%%%%%%%%%%%%%%%%%%%%%%%%%%%%%%%%%%%%%%%%%%%%%%%%%%%%%%%%
\setlength{\columnsep}{1.5em}
\setlength{\columnseprule}{0mm}

%%%%%%%%%%%%%%%%%%%%%%%%%%%%%%%%%%%%%%%%%%%%%%%%%%%%%%%%%%%%%%%%%%%%%%%%%%%%%%%%
% Save space in lists. Use this after the opening of the list
%%%%%%%%%%%%%%%%%%%%%%%%%%%%%%%%%%%%%%%%%%%%%%%%%%%%%%%%%%%%%%%%%%%%%%%%%%%%%%%%
\newcommand{\compresslist}{%
\setlength{\itemsep}{1pt}%
\setlength{\parskip}{0pt}%
\setlength{\parsep}{0pt}%
}

%%%%%%%%%%%%%%%%%%%%%%%%%%%%%%%%%%%%%%%%%%%%%%%%%%%%%%%%%%%%%%%%%%%%%%%%%%%%%%
%%% Begin of Document
%%%%%%%%%%%%%%%%%%%%%%%%%%%%%%%%%%%%%%%%%%%%%%%%%%%%%%%%%%%%%%%%%%%%%%%%%%%%%%

\begin{document}

%%%%%%%%%%%%%%%%%%%%%%%%%%%%%%%%%%%%%%%%%%%%%%%%%%%%%%%%%%%%%%%%%%%%%%%%%%%%%%
%%% Here starts the poster
%%%---------------------------------------------------------------------------
%%% Format it to your taste with the options
%%%%%%%%%%%%%%%%%%%%%%%%%%%%%%%%%%%%%%%%%%%%%%%%%%%%%%%%%%%%%%%%%%%%%%%%%%%%%%
% Define some colors

%\definecolor{lightblue}{cmyk}{0.83,0.24,0,0.12}
\definecolor{lightblue}{rgb}{0.145,0.6666,1}

% Draw a video
\newlength{\FSZ}
\newcommand{\drawvideo}[3]{% [0 0.25 0.5 0.75 1 1.25 1.5]
   \noindent\pgfmathsetlength{\FSZ}{\linewidth/#2}
   \begin{tikzpicture}[outer sep=0pt,inner sep=0pt,x=\FSZ,y=\FSZ]
   \draw[color=lightblue!50!black] (0,0) node[outer sep=0pt,inner sep=0pt,text width=\linewidth,minimum height=0] (video) {\noindent#3};
   \path [fill=lightblue!50!black,line width=0pt] 
     (video.north west) rectangle ([yshift=\FSZ] video.north east) 
    \foreach \x in {1,2,...,#2} {
      {[rounded corners=0.6] ($(video.north west)+(-0.7,0.8)+(\x,0)$) rectangle +(0.4,-0.6)}
    }
;
   \path [fill=lightblue!50!black,line width=0pt] 
     ([yshift=-1\FSZ] video.south west) rectangle (video.south east) 
    \foreach \x in {1,2,...,#2} {
      {[rounded corners=0.6] ($(video.south west)+(-0.7,-0.2)+(\x,0)$) rectangle +(0.4,-0.6)}
    }
;
   \foreach \x in {1,...,#1} {
     \draw[color=lightblue!50!black] ([xshift=\x\linewidth/#1] video.north west) -- ([xshift=\x\linewidth/#1] video.south west);
   }
   \foreach \x in {0,#1} {
     \draw[color=lightblue!50!black] ([xshift=\x\linewidth/#1,yshift=1\FSZ] video.north west) -- ([xshift=\x\linewidth/#1,yshift=-1\FSZ] video.south west);
   }
   \end{tikzpicture}
}

\hyphenation{resolution occlusions}
%%
\begin{poster}%
  % Poster Options
  {
  % Show grid to help with alignment
  grid=false,
  % Column spacing
  colspacing=1em,
  columns=2,
  % Color style
  bgColorOne=gray,
  bgColorTwo=white,
  borderColor=black,
  headerColorOne=white,
  headerColorTwo=lightblue,
  headerFontColor=black,
  boxColorOne=white,
  boxColorTwo=lightblue,
  % Format of textbox
  textborder=roundedleft,
  % Format of text header
  eyecatcher=true,
  headerborder=closed,
  headerheight=0.1\textheight,
 % textfont=\sc
  headershape=roundedright,
  headershade=shadelr,
  headerfont=\Large\textsc, %Sans Serif,
  textfont=\large{\setlength{\parindent}{1.5em}},
  boxshade=plain,
%  background=shade-tb,
  background=plain,
  linewidth=2pt
  }
  % Eye Catcher
  {\includegraphics[height=8em,keepaspectratio=true]{images/logo_uprrp}} 
  % Title
  {\bf\textsc{Number of Permutation Polynomials}\vspace{0.1em}}
  % Authors
  {\color{white}{\textsc{Christian A. Rodr\'{\i}guez; Alex D. Santos; Ivelisse Rubio; Francis Castro; \\ Department of Computer Science, \\ University of Puerto Rico, Rio Piedras Campus}}}
  % University logo
  {% The makebox allows the title to flow into the logo, this is a hack because of the L shaped logo.
    \includegraphics[height=8em,keepaspectratio=true]{images/logo_curm}
  }

%%%%%%%%%%%%%%%%%%%%%%%%%%%%%%%%%%%%%%%%%%%%%%%%%%%%%%%%%%%%%%%%%%%%%%%%%%%%%%
%%% Now define the boxes that make up the poster
%%%---------------------------------------------------------------------------
%%% Each box has a name and can be placed absolutely or relatively.
%%% The only inconvenience is that you can only specify a relative position 
%%% towards an already declared box. So if you have a box attached to the 
%%% bottom, one to the top and a third one which should be in between, you 
%%% have to specify the top and bottom boxes before you specify the middle 
%%% box.
%%%%%%%%%%%%%%%%%%%%%%%%%%%%%%%%%%%%%%%%%%%%%%%%%%%%%%%%%%%%%%%%%%%%%%%%%%%%%%
    %
    % A coloured circle useful as a bullet with an adjustably strong filling
    \newcommand{\colouredcircle}{%
      \tikz{\useasboundingbox (-0.2em,-0.32em) rectangle(0.2em,0.32em); \draw[draw=black,fill=lightblue,line width=0.03em] (0,0) circle(0.18em);}}

%%%%%%%%%%%%%%%%%%%%%%%%%%%%%%%%%%%%%%%%%%%%%%%%%%%%%%%%%%%%%%%%%%%%%%%%%%%%%%
  \headerbox{Abstract}{name=abstract,column=0,row=0}{
%%%%%%%%%%%%%%%%%%%%%%%%%%%%%%%%%%%%%%%%%%%%%%%%%%%%%%%%%%%%%%%%%%%%%%%%%%%%%%
 \vspace{0.3em}
Dado un trinomio de la forma $f_{a,b}(X)=X^r( X^{\frac{q-1}{d_1}}+aX^{\frac{q-1}{d_2}}+b )$ sobre un cuerpo finito $\mathbb{F}_q$ con tama\~{n}o de value set $s$, constru\'{\i}mos $d=lcm(d_1, d_2)$ otros trinomios en $\mathbb{F}_q$ con el mismo tama\~{n}o de value set. En particular, dado un polinomio de permutaci\'{o}n de la forma $f_{a,b}$, constru\'{\i}mos $d=lcm(d_1,d_2)$ otros polinomios de permutaci\'{o}n en $\mathbb{F}_q$. Tambi\'{e}n constru\'{\i}mos secuencias $P_{q^{m_1}}, P_{q^{m_2}}, \cdots,$ donde $P_{q^{m_i}}$ es un polinomio de permutaci\'{o}n en $\mathbb{F}_{q^{m_i}}$.
   
   \vspace{0.3em}
 }\label{Abstract}

%%%%%%%%%%%%%%%%%%%%%%%%%%%%%%%%%%%%%%%%%%%%%%%%%%%%%%%%%%%%%%%%%%%%%%%%%%%%%%
\headerbox{Problema}{name=problem,column=1,row=0}{
  %%%%%%%%%%%%%%%%%%%%%%%%%%%%%%%%%%%%%%%%%%%%%%%%%%%%%%%%%%%%%%%%%%%%%%%%%%%%%%
       \vspace{0.3em}
      {\Large Estudiar el value set de polinomios de la forma {\LARGE $$F_{a,b}(X) = X(X^{\frac{q-1}{d_1}} + aX^{\frac{q-1}{d_2}} +b)$$} sobre cuerpos finitos $\mathbb{F}_q$ y determinar condiciones en $a,b$ tal que el polinomio es un polinomio de permutaci\'on.}

      \vspace{0.3em}
}\label{Problem}

%%%%%%%%%%%%%%%%%%%%%%%%%%%%%%%%%%%%%%%%%%%%%%%%%%%%%%%%%%%%%%%%%%%%%%%%%%%%%%
  \headerbox{Resultados - Value Sets}{name=contribution,column=1,below=problem,above=bottom}{
%%%%%%%%%%%%%%%%%%%%%%%%%%%%%%%%%%%%%%%%%%%%%%%%%%%%%%%%%%%%%%%%%%%%%%%%%%%%%%

Definimos una relaci\'on para construir clases de equivalencia de polinomios con value sets de la misma cardinalidad.


\begin{definicion}\label{relaci\'on}

  Sean $a = \alpha^i, b = \alpha^j$, donde $\alpha$ es una raiz primitiva en $\mathbb{F}_q$, y $\sim$ una relaci\'on en $\mathbb{F}_q^* \times \mathbb{F}_q^*$ definida por: $(a,b) \sim (a', b')$ 

  $\Longleftrightarrow a' = \alpha^{i+h(\frac{q-1}{d_1} - \frac{q-1}{d_2})}, b' = \alpha^{j+h(\frac{q-1}{d_1})}$, donde $h \in \mathbb{Z}$.

\end{definicion}

  \begin{ejemplo*}
    Sean $q = 13, d_1 = 2, d_2 = 3$, entonces tenemos $\alpha = 2$ y $a = 2^2 = 4, b = 2^3 = 8$. Ahora $(a,b) \sim (a',b')$ si y solo si
    $a' = \alpha^{2+2h}, b' = \alpha^{3+6h}$. Por ejemplo $(2^2,2^3) \sim (2^{2+2},2^{3+6})$.
  \end{ejemplo*}

\begin{lema}
  
  La relaci\'on $\sim$ en Def~\ref{relaci\'on} en una relaci\'on de equivalencia en $\mathbb{F}_q^* \times \mathbb{F}_q^*$.

\end{lema}

  La relaci\'on de equivalencia definida anteriormente induce una relaci\'on de equivalencia en el conjunto de polinomios de la forma $F_{a,b}(X) = X(X^{\frac{q-1}{d_1}} + aX^{\frac{q-1}{d_2}} +b)$ con clases de equivalencia $[F_{a,b}] = [F_{\alpha^i, \alpha^j}] = \left\{\ F_{a',b'} | a' = \alpha^{i+h(\frac{q-1}{d_1} - \frac{q-1}{d_2})}, b' = \alpha^{j+h(\frac{q-1}{d_1})} \right\}$.

  Esto provee una construcci\'{o}n para polinomios con value sets de la misma cardinalidad.
\begin{teorema}
  
  Suponer que $F_{a,b} \sim F_{a',b'}$ donde $\sim$ es la relaci\'on de equivalencia en el Lema 1. Entonces $|V(F_{a,b})| = |V(F_{a',b'})|$.

\end{teorema}

\includegraphics[width=11cm, height=6cm]{clases-espa}

 % \begin{ejemplo*}
 %    Let $q = 13, d_1 = 2, d_2 = 3, a = 4, b = 8$. Since $(4,8) \sim (3,5)$ we have that $|V(F_{4, 8})| = |V(F_{3, 5})|$. In fact $V(F_{4, 8}) = \left\{0, 1, 2, 3, 10, 11, 12\right\}$, $V(F_{3, 5}) = \left\{0, 2, 4, 6, 7, 9, 11\right\}$. Note that even though the cardinalities are equal the value sets are not.
 %  \end{ejemplo*}
}\label{Results}

%%%%%%%%%%%%%%%%%%%%%%%%%%%%%%%%%%%%%%%%%%%%%%%%%%%%%%%%%%%%%%%%%%%%%%%%%%%%%%
  \headerbox{Preliminares}{name=preliminaries,column=0,below=abstract}{
%%%%%%%%%%%%%%%%%%%%%%%%%%%%%%%%%%%%%%%%%%%%%%%%%%%%%%%%%%%%%%%%%%%%%%%%%%%%%%
\vspace{0.3em}

\begin{definicion*}
  Una \textbf{permutaci\'on} de un conjunto $A$ es un ordenamiento de los elementos de $A$. Una funcion $f: A \rightarrow A$ nos da una permutaci\'on de $A$ si y solo si $f$ es uno a uno y sobre.
\end{definicion*}

\begin{definicion*}
  Un \textbf{cuerpo finito} $\mathbb{F}_{q}$, $q=p^r$, $p$ primo, es un conjunto con $q=p^r$ elementos.
\end{definicion*}

\begin{definicion*}
  Una \textbf{raiz primitiva} $\alpha \in \mathbb{F}_q$ es un generador del grupo multiplicativo $\mathbb{F}_{q}^{*}$.
\end{definicion*}

\begin{ejemplo*}
  Considere el cuerpo finito $\mathbb{F}_{7}$. Tenemos que: $3^1 = 3, 3^2 = 2, 3^3 = 6, 3^4 = 4, 3^5 = 5, 3^6 = 1$, entonces $3$ es una raiz primitiva de $\mathbb{F}_{7}$.
\end{ejemplo*}

\begin{definicion*}
  Sea $f(x)$ un polinomios definido sobre $\mathbb{F}_{q}$. El \textbf{value set} de $f$ esta definido por $V_{f} = \left\{f(a) \mid a \in \mathbb{F}_{q} \right\}$.
\end{definicion*}

Note que un polinomios $f(x)$ definido por $\mathbb{F}_{q}$ es un polinomio de permutaci\'on si y solo si $V_{f} = \mathbb{F}_{q}$.

   \vspace{0.3em}
}\label{Preliminaries}

%%%%%%%%%%%%%%%%%%%%%%%%%%%%%%%%%%%%%%%%%%%%%%%%%%%%%%%%%%%%%%%%%%%%%%%%%%%%%%
  \headerbox{Agradecimientos}{name=acknowledgement,column=0,below=motivation,above=bottom}{
%%%%%%%%%%%%%%%%%%%%%%%%%%%%%%%%%%%%%%%%%%%%%%%%%%%%%%%%%%%%%%%%%%%%%%%%%%%%%%

Este trabajo ha sido apoyado por una beca de el \textit{Center of Undergraduate Research in Matematics} (CURM) de \textit{Brigham Young University}.

}\label{Motivation}

\end{poster}

\end{document}
