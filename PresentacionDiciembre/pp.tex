

\documentclass{beamer}



\mode<presentation>
{
  \usetheme{Warsaw}


}


\usepackage[english]{babel}
% or whatever

\usepackage[utf8]{inputenc}
% or whatever

\usepackage{times}
\usepackage[T1]{fontenc}
% Or whatever. Note that the encoding and the font should match. If T1
% does not look nice, try deleting the line with the fontenc.

\usepackage{amsthm}
\usepackage{amssymb}
\usepackage{amsmath}



\newtheorem{proposition}{Proposition}
\newtheorem{conjecture}{Conjecture}
\newtheorem{ourproblem}{Our Problem}
\newtheorem*{definition*}{Definition}


\title
{Value Sets Of A Class Of Trinomials}

\author
{Christian A. Rodriguez\\
Alex D. Santos}

\institute[]
{
  Department of Computer Science\\
  University of Puerto Rico, Rio Piedras
}

\date
{\today}



% If you wish to uncover everything in a step-wise fashion, uncomment
% the following command: 

%\beamerdefaultoverlayspecification{<+->}

\begin{document}

\begin{frame}
  \titlepage
\end{frame}

\AtBeginSection[]
{
  \begin{frame}
    \frametitle{Table of Contents}
    \tableofcontents[currentsection]
  \end{frame}
}

\section{Introduction} % (fold)
\label{sec:introduction}



\begin{frame}{Polynomials in Finite Fields}

  A \textbf{finite field} $\mathbb{F}_{q}$, $q=p^r$, $p$ prime, is a field with $q=p^r$ elements.

  \begin{definition*}
    Let $f(x)$ be a polynomial defined over a finite field $\mathbb{F}_{q}$. This means that the domain of $f$ is equal to $\mathbb{F}_{q}$.
  \end{definition*}

  \begin{example}
  Consider the polynomial $f(x) = x+3$ defined over $\mathbb{F}_{5}$. We have that the domian of f is $\left\{0, 1, 2, 3, 4 \right\}$.
  \end{example}

\end{frame}

\begin{frame}{Value Sets}

\begin{definition*}
  Let $f(x)$ be a polynomial defined over a finite field $\mathbb{F}_{q}$. Then the \textbf{value set} of $f$ is defined as $V_{f} = \left\{f(a) \mid a \in \mathbb{F}_{q} \right\}$
\end{definition*}

\begin{example}
  Consider the polynomial $f(x) = x^2$ defined over $\mathbb{F}_{5}$. We have that $f(0) = 0, f(1) = 1, f(2) = 4, f(3) = 4, f(4) = 1$, so $V_{f} = \left\{0, 1, 4 \right\}$.
\end{example}

\end{frame}

\begin{frame}{Permutation Polynomials}

\begin{definition}
  A polynomial $f(x)$ defined over $\mathbb{F}_{q}$ is a permutation polynomial if and only if  $V_{f} = \mathbb{F}_{q}$.
\end{definition}


\begin{example}
  Consider the polynomial $f(x) = x+3$ defined over $\mathbb{F}_{7}$. We have that $f(0) = 3, f(1) = 4, f(2) = 5, f(3) = 6, f(4) = 0, f(5) = 1, f(6) = 2$, so $f(x)$ is a permutation polynomial over $\mathbb{F}_{7}$
\end{example}

Applications: 

\end{frame}


\begin{frame}{Primitive Roots}

\begin{definition*}
  A \textbf{primitive root} $\alpha \in \mathbb{F}_q$ is a generator for the multiplicative group $\mathbb{F}_{q}^{\times}$
\end{definition*}

\begin{example}
  Consider the finite field $\mathbb{F}_{7}$. We have that: $3^1 = 3, 3^2 = 2, 3^3 = 6, 3^4 = 4, 3^5 = 5, 3^6 = 1$, so $3$ is a primitive root of $\mathbb{F}_{7}$.
\end{example}

\end{frame}

% section introduction (end)

\section{Our Problem} % (fold)
\label{sec:our_problem}

% section our_problem (end)

\begin{frame}{Our Polynomial}
  

\end{frame}

\begin{frame}{The class of equivalence $(a,b)$}
  
  Definirla

\end{frame}

\begin{frame}{The class of equivalence $(a,b)$}
  
  Demostrar que es clase de equivalencia

\end{frame}

\begin{frame}{Problem}
  
\end{frame}

\section{Results} % (fold)
\label{sec:results}

\begin{frame}{Value set correspondence}
  Prop 1.4
\end{frame}

\begin{frame}{Size of equivalence classes}
  Prop 1.5
\end{frame}

\begin{frame}{Polynomials with Value sets of the same size}
  Prop 1.6 NO ESTA DEMOSTRADA EN EL PAPER
\end{frame}

\begin{frame}{Future Work}
  Conditions on $a,b$ that provide us with PP.
  Otras mas.
\end{frame}

% section results (end)


\end{document}


