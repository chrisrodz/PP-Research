

\documentclass{beamer}



\mode<presentation>
{
  \usetheme{Warsaw}


}


\usepackage[english]{babel}
% or whatever

\usepackage[utf8]{inputenc}
% or whatever

\usepackage{times}
\usepackage[T1]{fontenc}
% Or whatever. Note that the encoding and the font should match. If T1
% does not look nice, try deleting the line with the fontenc.

\usepackage{amsthm}
\usepackage{amssymb}
\usepackage{amsmath}



\newtheorem{proposition}{Proposition}
\newtheorem{conjecture}{Conjecture}
\newtheorem{ourproblem}{Our Problem}


\title
{Value Sets Of A Class Of Trinomials}

\author
{Christian A. Rodriguez\\
Alex D. Santos}

\institute[]
{
  Department of Computer Science\\
  University of Puerto Rico, Rio Piedras
}

\date
{\today}



% If you wish to uncover everything in a step-wise fashion, uncomment
% the following command: 

%\beamerdefaultoverlayspecification{<+->}

\begin{document}

\begin{frame}
  \titlepage
\end{frame}

\AtBeginSection[]
{
  \begin{frame}
    \frametitle{Table of Contents}
    \tableofcontents[currentsection]
  \end{frame}
}

\section{Introduction} % (fold)
\label{sec:introduction}



\begin{frame}{Polynomials in Finite Fields}


\end{frame}

\begin{frame}{Value Sets}


\end{frame}

\begin{frame}{Permutation Polynomials}


Applications: 

\end{frame}


\begin{frame}{Primitive Roots}


\end{frame}

% section introduction (end)

\section{Our Problem} % (fold)
\label{sec:our_problem}

% section our_problem (end)

\begin{frame}{Our Polynomial}
  
  \setbeamercovered{transparent}

  Let $d_1, d_2 \in \mathbb{F}_q$ such that $d_1 \mid q-1$ y $d_2 \mid q-1$. We are interested in the polynomial:
  {\Large$$F_{a,b}(x) = x(x^{\frac{q-1}{d_1}} + ax^{\frac{q-1}{d_2}} +b)$$ }
  with $a,b \in \mathbb{F}_q^{\times}$. 
  \pause 
  \linebreak
  Denote the value set of this polynomial $V_{a,b}$.

\end{frame}

\begin{frame}{The class of equivalence $(a,b)$}
  
  Let $a = \alpha^i, b = \alpha^j$ and $\sim$ the relation defined as $(a,b) \sim (a', b')$ 
  $<=> a' = \alpha^{i+h(\frac{q-1}{d_1} - \frac{q-1}{d_2})}, b' = \alpha^{j+h(\frac{q-1}{d_1})}$

  \alert{Example}

\end{frame}

\begin{frame}{The class of equivalence $(a,b)$}
  
  \begin{proposition}
    The relation $\sim$ defined above is an equivalence relation.
  \end{proposition}

\end{frame}

\begin{frame}{Problem}
  \begin{ourproblem}
    Study the value set of polynomials of the form $F_{a,b}(x) = x(x^{\frac{q-1}{d_1}} + ax^{\frac{q-1}{d_2}} +b)$ and determine conditions in $a,b$ such that they are permutation polynomials.
  \end{ourproblem}
\end{frame}

\section{Results} % (fold)
\label{sec:results}

\begin{frame}{Value set correspondence}
  \begin{proposition}

    Let $[a, b]$ be the class of equivalence of $(a, b)$. If $(a', b') \in [a, b]$,
     then $|V_{a', b'}| = |V_{a, b}|$.

  \end{proposition}
\end{frame}

\begin{frame}{Size of equivalence classes}
  
  \begin{proposition}
    $|[a, b]| = lcm(d_1,d_2)$ where $lcm(x,y)$ is the least common multiple of $x$ and $y$.
  \end{proposition}
\end{frame}

\begin{frame}{Polynomials with Value sets of the same size}
  \begin{proposition}
    The number of polynomials of the form $F_{a, b}(x)$ with $|V_{a, b}| = n$ is a multiple of $|[a, b]|$
  \end{proposition}
\end{frame}

\begin{frame}{Future Work}
  \begin{itemize}
    \item Find necessary and sufficient conditions such that $V_{a,b} = \mathbb{F}_q$
    \item Study our results on the family of polynomials of the form $F_{a,b}(x) = x^m(x^{\frac{q-1}{d_1}} + ax^{\frac{q-1}{d_2}} +b)$
  \end{itemize}
\end{frame}

% section results (end)


\end{document}


