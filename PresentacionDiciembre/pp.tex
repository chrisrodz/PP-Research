

\documentclass{beamer}



\mode<presentation>
{
  \usetheme{Warsaw}


}


\usepackage[english]{babel}
% or whatever

\usepackage[utf8]{inputenc}
% or whatever

\usepackage{times}
\usepackage[T1]{fontenc}
% Or whatever. Note that the encoding and the font should match. If T1
% does not look nice, try deleting the line with the fontenc.

\usepackage{amsthm}
\usepackage{amssymb}
\usepackage{amsmath}



\newtheorem{proposition}{Proposition}
\newtheorem{conjecture}{Conjecture}
\newtheorem{ourproblem}{Our Problem}


\title
{Value Sets Of A Class Of Trinomials}

\author
{Christian A. Rodriguez\\
Alex D. Santos}

\institute[]
{
  Department of Computer Science\\
  University of Puerto Rico, Rio Piedras
}

\date
{\today}



% If you wish to uncover everything in a step-wise fashion, uncomment
% the following command: 

%\beamerdefaultoverlayspecification{<+->}

\begin{document}

\begin{frame}
  \titlepage
\end{frame}

\AtBeginSection[]
{
  \begin{frame}
    \frametitle{Table of Contents}
    \tableofcontents[currentsection]
  \end{frame}
}

\section{Introduction} % (fold)
\label{sec:introduction}



\begin{frame}{Polynomials in Finite Fields}


\end{frame}

\begin{frame}{Value Sets}


\end{frame}

\begin{frame}{Permutation Polynomials}


Applications: 

\end{frame}


\begin{frame}{Primitive Roots}


\end{frame}

% section introduction (end)

\section{Our Problem} % (fold)
\label{sec:our_problem}

% section our_problem (end)

\begin{frame}{Our Polynomial}
  

\end{frame}

\begin{frame}{The class of equivalence $(a,b)$}
  
  Definirla

\end{frame}

\begin{frame}{The class of equivalence $(a,b)$}
  
  Demostrar que es clase de equivalencia

\end{frame}

\begin{frame}{Problem}
  
\end{frame}

\section{Results} % (fold)
\label{sec:results}

\begin{frame}{Value set correspondence}
  Prop 1.4
\end{frame}

\begin{frame}{Size of equivalence classes}
  Prop 1.5
\end{frame}

\begin{frame}{Polynomials with Value sets of the same size}
  Prop 1.6 NO ESTA DEMOSTRADA EN EL PAPER
\end{frame}

\begin{frame}{Future Work}
  Conditions on $a,b$ that provide us with PP.
  Otras mas.
\end{frame}

% section results (end)


\end{document}


