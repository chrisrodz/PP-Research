\documentclass[landscape,final,paperwidth=48in,paperheight=38in]{baposter}

\usepackage{calc}
\usepackage{graphicx}
\usepackage{amsmath}
\usepackage{amssymb}
\usepackage{relsize}
\usepackage{multirow}
\usepackage{rotating}
\usepackage{bm}
\usepackage{url} 

\usepackage{amsthm} 
\usepackage{amssymb}
\usepackage{amsmath}

\newtheorem{corollary}{Corollary}
\newtheorem{conjecture}{Conjecture}
\newtheorem{example}{Example}
\newtheorem{definition}{Definition}
\newtheorem{proposition}{Proposition}
\newtheorem{lemma}{Lemma}
\newtheorem{theorem}{Theorem}

\usepackage{graphicx}
\usepackage{multicol}

%\usepackage{times}
%\usepackage{helvet}
%\usepackage{bookman}
\usepackage{palatino}

\newcommand{\captionfont}{\footnotesize}

\graphicspath{{images/}{../images/}}
\usetikzlibrary{calc}

\newcommand{\SET}[1]  {\ensuremath{\mathcal{#1}}}
\newcommand{\MAT}[1]  {\ensuremath{\boldsymbol{#1}}}
\newcommand{\VEC}[1]  {\ensuremath{\boldsymbol{#1}}}
\newcommand{\Video}{\SET{V}}
\newcommand{\video}{\VEC{f}}
\newcommand{\track}{x}
\newcommand{\Track}{\SET T}
\newcommand{\LMs}{\SET L}
\newcommand{\lm}{l}
\newcommand{\PosE}{\SET P}
\newcommand{\posE}{\VEC p}
\newcommand{\negE}{\VEC n}
\newcommand{\NegE}{\SET N}
\newcommand{\Occluded}{\SET O}
\newcommand{\occluded}{o}

%%%%%%%%%%%%%%%%%%%%%%%%%%%%%%%%%%%%%%%%%%%%%%%%%%%%%%%%%%%%%%%%%%%%%%%%%%%%%%%%
%%%% Some math symbols used in the text
%%%%%%%%%%%%%%%%%%%%%%%%%%%%%%%%%%%%%%%%%%%%%%%%%%%%%%%%%%%%%%%%%%%%%%%%%%%%%%%%

%%%%%%%%%%%%%%%%%%%%%%%%%%%%%%%%%%%%%%%%%%%%%%%%%%%%%%%%%%%%%%%%%%%%%%%%%%%%%%%%
% Multicol Settings
%%%%%%%%%%%%%%%%%%%%%%%%%%%%%%%%%%%%%%%%%%%%%%%%%%%%%%%%%%%%%%%%%%%%%%%%%%%%%%%%
\setlength{\columnsep}{1.5em}
\setlength{\columnseprule}{0mm}

%%%%%%%%%%%%%%%%%%%%%%%%%%%%%%%%%%%%%%%%%%%%%%%%%%%%%%%%%%%%%%%%%%%%%%%%%%%%%%%%
% Save space in lists. Use this after the opening of the list
%%%%%%%%%%%%%%%%%%%%%%%%%%%%%%%%%%%%%%%%%%%%%%%%%%%%%%%%%%%%%%%%%%%%%%%%%%%%%%%%
\newcommand{\compresslist}{%
\setlength{\itemsep}{1pt}%
\setlength{\parskip}{0pt}%
\setlength{\parsep}{0pt}%
}

%%%%%%%%%%%%%%%%%%%%%%%%%%%%%%%%%%%%%%%%%%%%%%%%%%%%%%%%%%%%%%%%%%%%%%%%%%%%%%
%%% Begin of Document
%%%%%%%%%%%%%%%%%%%%%%%%%%%%%%%%%%%%%%%%%%%%%%%%%%%%%%%%%%%%%%%%%%%%%%%%%%%%%%

\begin{document}

%%%%%%%%%%%%%%%%%%%%%%%%%%%%%%%%%%%%%%%%%%%%%%%%%%%%%%%%%%%%%%%%%%%%%%%%%%%%%%
%%% Here starts the poster
%%%---------------------------------------------------------------------------
%%% Format it to your taste with the options
%%%%%%%%%%%%%%%%%%%%%%%%%%%%%%%%%%%%%%%%%%%%%%%%%%%%%%%%%%%%%%%%%%%%%%%%%%%%%%
% Define some colors

%\definecolor{lightblue}{cmyk}{0.83,0.24,0,0.12}
\definecolor{lightblue}{rgb}{0.145,0.6666,1}

% Draw a video
\newlength{\FSZ}
\newcommand{\drawvideo}[3]{% [0 0.25 0.5 0.75 1 1.25 1.5]
   \noindent\pgfmathsetlength{\FSZ}{\linewidth/#2}
   \begin{tikzpicture}[outer sep=0pt,inner sep=0pt,x=\FSZ,y=\FSZ]
   \draw[color=lightblue!50!black] (0,0) node[outer sep=0pt,inner sep=0pt,text width=\linewidth,minimum height=0] (video) {\noindent#3};
   \path [fill=lightblue!50!black,line width=0pt] 
     (video.north west) rectangle ([yshift=\FSZ] video.north east) 
    \foreach \x in {1,2,...,#2} {
      {[rounded corners=0.6] ($(video.north west)+(-0.7,0.8)+(\x,0)$) rectangle +(0.4,-0.6)}
    }
;
   \path [fill=lightblue!50!black,line width=0pt] 
     ([yshift=-1\FSZ] video.south west) rectangle (video.south east) 
    \foreach \x in {1,2,...,#2} {
      {[rounded corners=0.6] ($(video.south west)+(-0.7,-0.2)+(\x,0)$) rectangle +(0.4,-0.6)}
    }
;
   \foreach \x in {1,...,#1} {
     \draw[color=lightblue!50!black] ([xshift=\x\linewidth/#1] video.north west) -- ([xshift=\x\linewidth/#1] video.south west);
   }
   \foreach \x in {0,#1} {
     \draw[color=lightblue!50!black] ([xshift=\x\linewidth/#1,yshift=1\FSZ] video.north west) -- ([xshift=\x\linewidth/#1,yshift=-1\FSZ] video.south west);
   }
   \end{tikzpicture}
}

\hyphenation{resolution occlusions}
%%
\begin{poster}%
  % Poster Options
  {
  % Show grid to help with alignment
  grid=false,
  % Column spacing
  colspacing=1em,
  % Color style
  bgColorOne=white,
  bgColorTwo=white,
  borderColor=lightblue,
  headerColorOne=black,
  headerColorTwo=lightblue,
  headerFontColor=white,
  boxColorOne=white,
  boxColorTwo=lightblue,
  % Format of textbox
  textborder=roundedleft,
  % Format of text header
  eyecatcher=true,
  headerborder=closed,
  headerheight=0.1\textheight,
%  textfont=\sc, An example of changing the text font
  headershape=roundedright,
  headershade=shadelr,
  headerfont=\Large\bf\textsc, %Sans Serif
  textfont={\setlength{\parindent}{1.5em}},
  boxshade=plain,
%  background=shade-tb,
  background=plain,
  linewidth=2pt
  }
  % Eye Catcher
  {\includegraphics[height=8em,keepaspectratio=true]{images/logo_uprrp}} 
  % Title
  {\bf\textsc{On a Class of Permutation Polynomials}\vspace{0.1em}}
  % Authors
  {\textsc{Christian A. Rodr\'{\i}guez \& Alex D. Santos \\ University of Puerto Rico, Rio Piedras \\ Department of Computer Science}}
  % University logo
  {% The makebox allows the title to flow into the logo, this is a hack because of the L shaped logo.
    \includegraphics[height=9em,keepaspectratio=true]{images/logo_ccom}
  }

%%%%%%%%%%%%%%%%%%%%%%%%%%%%%%%%%%%%%%%%%%%%%%%%%%%%%%%%%%%%%%%%%%%%%%%%%%%%%%
%%% Now define the boxes that make up the poster
%%%---------------------------------------------------------------------------
%%% Each box has a name and can be placed absolutely or relatively.
%%% The only inconvenience is that you can only specify a relative position 
%%% towards an already declared box. So if you have a box attached to the 
%%% bottom, one to the top and a third one which should be in between, you 
%%% have to specify the top and bottom boxes before you specify the middle 
%%% box.
%%%%%%%%%%%%%%%%%%%%%%%%%%%%%%%%%%%%%%%%%%%%%%%%%%%%%%%%%%%%%%%%%%%%%%%%%%%%%%
    %
    % A coloured circle useful as a bullet with an adjustably strong filling
    \newcommand{\colouredcircle}{%
      \tikz{\useasboundingbox (-0.2em,-0.32em) rectangle(0.2em,0.32em); \draw[draw=black,fill=lightblue,line width=0.03em] (0,0) circle(0.18em);}}

%%%%%%%%%%%%%%%%%%%%%%%%%%%%%%%%%%%%%%%%%%%%%%%%%%%%%%%%%%%%%%%%%%%%%%%%%%%%%%
  \headerbox{Abstract}{name=abstract,column=0,row=0}{
%%%%%%%%%%%%%%%%%%%%%%%%%%%%%%%%%%%%%%%%%%%%%%%%%%%%%%%%%%%%%%%%%%%%%%%%%%%%%%
   A polynomial $f(x)$ defined over a set $A$ is called a \textbf{permutation polynomial} if $f(x)$ acts as a permutation over the elements of $A$. This is, if $f: A \rightarrow A$ is 1-1 and onto. We are studying the coefficients $a$ and $b$ that make polynomials of the form $F_{a,b}(x)=x^{\frac{p+1}{2}} + ax^{\frac{p+5}{6}} + bx$ a permutation polynomial where $a,b \in \mathbb{F}_{q}^{\times}$. More specifically we study the family of polynomials: $F_{a,b}(x)=x^{\frac{p+1}{2}} + ax^{\frac{p+5}{6}} + bx$. Our approach in studying $F(x)$ is to use the division algorithm to consider $x=\alpha^{n}$ where $n=6k+r, r=0,...,5$. If $F_{a,b}(x)$ is a permutation, this partitions $\mathbb{F}_{q}^{\times}$ into 6 classes: $F_{a,b}(\alpha^{6k+r})$ for $r=0,...,5$.
   \vspace{0.3em}
 }\label{Abstract}

%%%%%%%%%%%%%%%%%%%%%%%%%%%%%%%%%%%%%%%%%%%%%%%%%%%%%%%%%%%%%%%%%%%%%%%%%%%%%%
  \headerbox{Value set of a class of polynomials}{name=contribution,column=1,row=0,span=2}{
%%%%%%%%%%%%%%%%%%%%%%%%%%%%%%%%%%%%%%%%%%%%%%%%%%%%%%%%%%%%%%%%%%%%%%%%%%%%%%
  \begin{multicols}{2}
    Our interest is studying the value set of a specific class of permutation polynomials. The class of polynomials we consider is defined as follows:
    $$F_{a,b}(x) = x^{\frac{q+1}{2}} + a\cdot x^{\frac{q+1}{d}} + b\cdot x$$
    
    Where $a,b \in \mathbb{F}_{q}$ and $d \mid q-1$. More formally, we would like to characterize the value set $V_{F}$ of $F_{a,b}(x)$ based on the parameters $a$ and $b$. It is easy to see that $F_{a,b}(0) = 0 \ \forall a,b \in \mathbb{F}_{q}$, it follows that $0$ is always in $V_{F}$. For a fixed pair $a,b$ we separate $V_{F} \setminus \left\{0\right\}$ into smaller subsets in the following way:

    \begin{definition}
      Let $F_{a,b}(x) = x^{\frac{q+1}{2}} + a\cdot x^{\frac{q+1}{d}} + b\cdot x$ be a polynomial defined over $\mathbb{F}_{q}$ where $d \mid q-1$. We define the sets $A_i = \left\{F_{a,b}(\alpha^{d\cdot k+i}) \mid k=0,...,\frac{q-1}{d}\right\}$ for $i=0,...,d-1$, where $\alpha$ is a primitive root of $\mathbb{F}_{q}$.
    \end{definition}

    Using properties of these sets we will characterize $V_{F}$. First we would like to note that for $i \neq j$ the sets $A_i$ and $A_j$ are either equal, or distinct.

    \begin{lemma}\label{conjuntos_disjuntos}
      Let $F_{a,b}(x)$ be defined over $\mathbb{F}_{q}$. For two sets $A_i$ and $A_j$ we must have that either $A_i \cap A_j = \emptyset$ or $A_i = A_j$.
    \end{lemma}

    Lemma~\ref{conjuntos_disjuntos} provides an immediate characterization of the value set and insight on conditions to make $F_{a,b}(x)$ a permutation polynomial. In our studies we also determine the size of the sets $A_i$.

    \begin{lemma}\label{tamanos_conjuntos}
      Let $F_{a,b}(x)$ be defined over $\mathbb{F}_{q}$ and $A_i$ be defined as above. We have that $\left\vert A_i \right\vert = \frac{q-1}{d}$ or $A_i = \left\{ 0 \right\}$
    \end{lemma}

    Now we are also interested in correlations between the pairs $a,b$ and the value sets of distinct polynomials of the for $F_{a,b}(x)$. We proved a lemma that gives us a correspondence among some of these polynomials. In other words, these polynomials have the same value set.

    \begin{lemma}\label{correspondencia}
      Let $F_{a,b}(x)$ defined over $\mathbb{F}_{q}$ and let $\alpha$ denote a primitive root of $\mathbb{F}_{q}$. If we write $a = \alpha^i$ and $b = \alpha^j$ then we have that
      $$F_{\alpha^i,\alpha^j}(\alpha^k) = -\alpha \cdot F_{\alpha^{i+(d+2)\cdot \frac{q-1}{2d}}, \alpha^{j+\frac{q-1}{2}}}(\alpha^{k-1})$$
    \end{lemma}

    From lemma~\ref{conjuntos_disjuntos} we know that for a fixed polynomial the sets $A_i$ are either distinct or equal. Finally from lemma~\ref{correspondencia} we have that up to $2d$ distinct polynomials of the form $F_{a,b}(x)$ have the same value set. This information gives us the following theorem:

    \begin{proposition}\label{el_teorema}
      Let $F_{a,b}(x)$ be defined over $\mathbb{F}_{q}$. Then we have that the amount of polynomials of the form $F_{a,b}(x)$ such that $\left\vert V_{F} \right\vert = r\cdot \frac{q-1}{d} + 1, r \leq d$ is divisible by $2d$ when $d$ is odd and by $d$ otherwise.
    \end{proposition}

  \end{multicols}

   \vspace{0.3em}
  }\label{Value sets}

%%%%%%%%%%%%%%%%%%%%%%%%%%%%%%%%%%%%%%%%%%%%%%%%%%%%%%%%%%%%%%%%%%%%%%%%%%%%%%
\headerbox{Applications}{name=applications,column=3,row=0}{
  %%%%%%%%%%%%%%%%%%%%%%%%%%%%%%%%%%%%%%%%%%%%%%%%%%%%%%%%%%%%%%%%%%%%%%%%%%%%%%
    \drawvideo{5}{40}{%
      \includegraphics[width=0.2\linewidth]{red-4-sec_000_rastered}%
        \includegraphics[width=0.2\linewidth]{red-4-sec_024_rastered}%
        \includegraphics[width=0.2\linewidth]{red-4-sec_048_rastered}%
        \includegraphics[width=0.2\linewidth]{red-4-sec_072_rastered}%
        \includegraphics[width=0.2\linewidth]{red-4-sec_095_rastered}%
    }
  \begin{tabular*}{\linewidth}{*{5}{@{}p{0.2\linewidth}@{}}}
    {\hfill{}Frame 0\hfill{}} & {\hfill{}24\hfill{}} &  {\hfill{}48\hfill{}} &  {\hfill{}72\hfill{}} & {\hfill{}95\hfill{}}
  \end{tabular*}
  \\[1em]
      \drawvideo{5}{40}{%
        \includegraphics[width=0.2\linewidth]{giraffe-run-000-rastered}%
          \includegraphics[width=0.2\linewidth]{giraffe-run-100-rastered}%
          \includegraphics[width=0.2\linewidth]{giraffe-run-200-rastered}%
          \includegraphics[width=0.2\linewidth]{giraffe-run-300-rastered}%
          \includegraphics[width=0.2\linewidth]{giraffe-run-458-rastered}%
      }
  \begin{tabular*}{\linewidth}{*{5}{@{}p{0.2\linewidth}@{}}}
    {\hfill{}Frame 0\hfill{}}  &
    {\hfill{}100\hfill{}} &
    {\hfill{}200\hfill{}} &
    {\hfill{}300\hfill{}} &
    {\hfill{}458\hfill{}} 
  \end{tabular*}
      \begin{multicols}{2}
    Between one and three user clicks were needed to achieve accurate tracking for
      the head sequence. Note the correct handling of the occluded ear, which
      required only a single click. 

      The eye of the running giraffe required eight user interactions, of which three
      marked occlusions. 
      \end{multicols}
      \vspace{-0.6em}
}\label{Applications}

%%%%%%%%%%%%%%%%%%%%%%%%%%%%%%%%%%%%%%%%%%%%%%%%%%%%%%%%%%%%%%%%%%%%%%%%%%%%%%
  \headerbox{Preliminaries}{name=preliminaries,column=0,below=abstract,above=bottom}{
%%%%%%%%%%%%%%%%%%%%%%%%%%%%%%%%%%%%%%%%%%%%%%%%%%%%%%%%%%%%%%%%%%%%%%%%%%%%%%

We are interested in studying sets known as Finite Fields.

\begin{definition}
  A \textbf{Finite Field} $\mathbb{F}_{q}$ is a field with $q=p^r$ elements, where $p$ is a prime number.
\end{definition}

An important property of finite fields is the existence of a primitive root, a generator of the nonzero elements of $\mathbb{F}_q$.

\begin{definition}
  A \textbf{primitive root} $\alpha \in \mathbb{F}_q$ is a generator for the multiplicative group $\mathbb{F}_{q}^{\times}$
\end{definition}

\begin{example}
  Consider the finite field $\mathbb{F}_{7}$. We have that: $3^1 = 3, 3^2 = 2, 3^3 = 6, 3^4 = 4, 3^5 = 5, 3^6 = 1$, so $3$ is a primitive root of $\mathbb{F}_{7}$.
\end{example}

We are interested in studying polynomials defined over finite fields. Specifically, our interest lies in the value set of these polynomials.

\begin{definition}
  Let $f(x)$ be a polynomial defined over a finite field $\mathbb{F}_{q}$. Then the \textbf{value set} of $f$ is defined as $V_{f} = \left\{f(a) \mid a \in \mathbb{F}_{q} \right\}$
\end{definition}

In our work we characterize $V_{f}$ for a specific class of polynomials defined over finite fields. From this characterization we can provide information on the amount of permutation polynomials of our class.

\begin{definition}
  Consider a finite field $\mathbb{F}_{q}$. A polynomial $f(x)$ defined over $\mathbb{F}_{q}$ is said to be a permutation polynomial if $V_{f} = \mathbb{F}_{q}$.
\end{definition}

\begin{example}
  Consider the polynomial $f(x) = x+3$ defined over $\mathbb{F}_{7}$. We have that $f(0) = 3, f(1) = 4, f(2) = 5, f(3) = 6, f(4) = 0, f(5) = 1, f(6) = 2$, so $f(x)$ is a permutation polynomial over $\mathbb{F}_{7}$
\end{example}

   \vspace{0.3em}
}\label{Preliminaries}

%%%%%%%%%%%%%%%%%%%%%%%%%%%%%%%%%%%%%%%%%%%%%%%%%%%%%%%%%%%%%%%%%%%%%%%%%%%%%%
  \headerbox{Conditions for permutations of the from $F_{a,b}(x)$}{name=questions,column=1,span=2,above=bottom,below=contribution}{
%%%%%%%%%%%%%%%%%%%%%%%%%%%%%%%%%%%%%%%%%%%%%%%%%%%%%%%%%%%%%%%%%%%%%%%%%%%%%%
  \begin{multicols}{2}
    Permutation polynomials over finite fields are polynomials whose value set is equal to the field. Using our previous results we present work on when the family of polynomials of the form $F_{a,b}(x)$ is a permutation polynomial.

    If we define the value set $V_{F}$ in terms of the sets $A_i$ then it follows from lemma~\ref{conjuntos_disjuntos} that all of the elements between these sets should be distinct for $F$ to be a permutation polynomial.

    \begin{lemma}\label{PP_disjuntos}
      Let $F_{a,b}(x)$ be a polynomial defined over $\mathbb{F}_{q}$ and $A_i$ the sets defined above. Then $F_{a,b}(x)$ is a permutation polynomial if and only if $A_i \cap A_j = \emptyset \ \forall\ i \neq j$ and $A_i \neq \left\{ 0 \right\} \ \forall \ i$
    \end{lemma}

    \begin{example}
      Lorem ipsum dolor sit amet, consectetur adipisicing elit, sed do eiusmod
      tempor incididunt ut labore et dolore magna aliqua. Ut enim ad minim veniam,
      quis nostrud exercitation ullamco laboris nisi ut aliquip ex ea commodo
      consequat. Duis aute irure dolor in reprehenderit in voluptate velit esse
      cillum dolore eu fugiat nulla pariatur. Excepteur sint occaecat cupidatat non
      proident, sunt in culpa qui officia deserunt mollit anim id est laborum.
    \end{example}

    Also, a particular case of proposition~\ref{el_teorema} is the case when $r=d$ and $\left\vert V_{F} \right\vert = q$, or $V_{F} = \mathbb{F}_{q}$. This case is exactly when $F_{a,b}(x)$ is a permutation polynomial over $\mathbb{F}_{q}$. And so it is easy to see that corollary~\ref{cantidad_pp} follows easily from proposition~\ref{el_teorema}.

    \begin{corollary}\label{cantidad_pp}
      The amount of permutation polynomials of the form $F_{a,b}(x) = x^{\frac{q+1}{2}} + a\cdot x^{\frac{q+1}{d}} + b\cdot x$ over $\mathbb{F}_{q}$ where $d \mid q-1$ is divisible by $2d$ for $d$ odd and by $d$ otherwise.
    \end{corollary}

    \begin{example}
      Lorem ipsum dolor sit amet, consectetur adipisicing elit, sed do eiusmod
      tempor incididunt ut labore et dolore magna aliqua. Ut enim ad minim veniam,
      quis nostrud exercitation ullamco laboris nisi ut aliquip ex ea commodo
      consequat. Duis aute irure dolor in reprehenderit in voluptate velit esse
      cillum dolore eu fugiat nulla pariatur. Excepteur sint occaecat cupidatat non
      proident, sunt in culpa qui officia deserunt mollit anim id est laborum.
    \end{example}

  \end{multicols}
   \vspace{0.3em}
  }\label{Conditions for PP}

%%%%%%%%%%%%%%%%%%%%%%%%%%%%%%%%%%%%%%%%%%%%%%%%%%%%%%%%%%%%%%%%%%%%%%%%%%%%%%
  \headerbox{Future Work}{name=future work,column=3,below=applications}{
%%%%%%%%%%%%%%%%%%%%%%%%%%%%%%%%%%%%%%%%%%%%%%%%%%%%%%%%%%%%%%%%%%%%%%%%%%%%%%
\noindent\begin{tabular}{r@{\hspace{0.3em}}c@{\hspace{1.5em}}c@{}}
& With & Without\\
& background model & background model\\
\begin{sideways}{\makebox[0.37\linewidth][c]{Ridge of the lips}}\end{sideways} &
\includegraphics[width=0.40\linewidth]{candidates_lips_ridge_left_bg} &
\includegraphics[width=0.40\linewidth]{candidates_lips_ridge_left_no_bg} \\[2em]
%\begin{sideways}{\makebox[0.32\linewidth][c]{Flank of a giraffe}}\end{sideways} & 
%\includegraphics[width=0.40\linewidth]{candidates_giraffes_flank_bg}&
%\includegraphics[width=0.40\linewidth]{candidates_giraffes_flank_no_bg}\\
\end{tabular}
   \vspace{0.3em}
  }\label{Future Work}

%%%%%%%%%%%%%%%%%%%%%%%%%%%%%%%%%%%%%%%%%%%%%%%%%%%%%%%%%%%%%%%%%%%%%%%%%%%%%%
  \headerbox{References}{name=applications,column=3,above=bottom,below=future work}{
%%%%%%%%%%%%%%%%%%%%%%%%%%%%%%%%%%%%%%%%%%%%%%%%%%%%%%%%%%%%%%%%%%%%%%%%%%%%%%
  The source code and compiled executables with an interactive interface are available at \\
  \url{http://www.cs.unibas.ch/personen/amberg_brian/graphtrack}
   \vspace{0.3em}
  }\label{References}

\end{poster}

\end{document}
