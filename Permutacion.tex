\documentclass[12pt]{article}

\usepackage{amsthm}
\usepackage{amssymb}

\begin{document}


Let $p \equiv 1 \bmod{3}$. Let $F(X)=X^{\frac{p+1}{2}}+aX^{\frac{p+5}{6}}+bX$ be a polynomial over $\mathbb{F}_{p}$. Let $\alpha$ be a primitive root in $\mathbb{F}_{p}$.
\\
Notice that $F(X)=X(X^{\frac{p-1}{2}}+aX^{\frac{p-1}{6}}+b)$. We will use the approach of considering the $\alpha^{6k+r},r=0,...,5$. This divides $F(X)$ into 6 classes:

\begin{itemize}
	\item
		$F(\alpha^{6k}) = \alpha^{6k}(1+a+b)$
	\item
		$F(\alpha^{6k+1})= \alpha^{6k+1}(1+a\alpha^{\frac{p-1}{6}}+b)$
	\item
		$F(\alpha^{6k+2})=\alpha^{6k+2}(1+a\alpha^{\frac{p-1}{3}}+b)$
	\item
		$F(\alpha^{6k+3})=\alpha^{6k+3}(-1-a+b)$
	\item
		$F(\alpha^{6k+4})=\alpha^{6k+4}(1+a\alpha^{2\frac{p-1}{3}}+b)$
	\item
		$F(\alpha^{6k+5})=\alpha^{6k+5}(1+a\alpha^{5\frac{p-1}{6}}+b)$
\end{itemize}

These classes should be disjoint, to find conditions on $a$ and $b$ we equate these classes (with different exponents of $6$).
\\

\textbf{Class $6k$}

\begin{itemize}
	\item Class $6l+1$ \\ \\
		$\alpha^{6k}(1+a+b) = \alpha^{6l+1}(1+a\alpha^{\frac{p-1}{6}}+b)$
		\\ \\
		$\alpha^{6(k-l)}(1+a+b) = \alpha(1+a\alpha^{\frac{p-1}{6}}+b)$
		\\ \\
		$\alpha^{6(k-l)} =  \alpha \frac{(1+a\alpha^{\frac{p-1}{6}}+b)}{(1+a+b)} $
		\\ \\
		From this we know that $a$ and $b$ must satisfy
		 $\alpha^{6m} \neq  \alpha \frac{(1+a\alpha^{\frac{p-1}{6}}+b)}{(1+a+b)} $

	\item Class $6l+2$ \\ \\
		$\alpha^{6k}(1+a+b) = \alpha^{6l+2}(1+a\alpha^{\frac{p-1}{3}}+b)$
		\\ \\
		$\alpha^{6(k-l)}(1+a+b) = \alpha^{2}(1+a\alpha^{\frac{p-1}{3}}+b)$
		\\ \\
		$\alpha^{6(k-l)} = \alpha^{2}\frac{(1+a\alpha^{\frac{p-1}{3}}+b)}{(1+a+b)}$
		\\ \\
		From this we know that $a$ and $b$ must satisfy
		$\alpha^{6m} \neq \alpha^{2}\frac{(1+a\alpha^{\frac{p-1}{3}}+b)}{(1+a+b)}$

	\item Class $6l+3$ \\ \\
		$\alpha^{6k}(1+a+b) = \alpha^{6l+3}(-1-a+b)$
		\\ \\
		$\alpha^{6(k-l)}(1+a+b) = \alpha^{3}(-1-a+b)$
		\\ \\
		$\alpha^{6(k-l)} = \alpha^{3}\frac{(-1-a+b)}{(1+a+b)}$
		\\ \\
		From this we know that $a$ and $b$ must satisfy
		$\alpha^{6m} \neq \alpha^{3}\frac{(-1-a+b)}{(1+a+b)}$
	
	\item Class $6l+4$ \\ \\
		$\alpha^{6k}(1+a+b) = \alpha^{6l+4}(1+a\alpha^{2\frac{p-1}{3}}+b)$
		\\ \\
		$\alpha^{6(k-l)}(1+a+b) = \alpha^{4}(1+a\alpha^{2\frac{p-1}{3}}+b)$
		\\ \\
		$\alpha^{6(k-l)} = \alpha^{4}\frac{(1+a\alpha^{2\frac{p-1}{3}}+b)}{(1+a+b)}$
		\\ \\
		From this we know that $a$ and $b$ must satisfy $\alpha^{6m} \neq \alpha^{4}\frac{(1+a\alpha^{2\frac{p-1}{3}}+b)}{(1+a+b)}$

	\item Class $6l+5$ \\ \\
		$\alpha^{6k}(1+a+b) = \alpha^{6l+5}(1+a\alpha^{5\frac{p-1}{6}}+b)$
		\\ \\
		$\alpha^{6(k-l)}(1+a+b) = \alpha^{5}(1+a\alpha^{5\frac{p-1}{6}}+b)$
		\\ \\
		$\alpha^{6(k-l)} = \alpha^{5}\frac{(1+a\alpha^{5\frac{p-1}{6}}+b)}{(1+a+b)}$
		\\ \\
		From this we know that $a$ and $b$ must satisfy $\alpha^{6m} \neq \alpha^{5}\frac{(1+a\alpha^{5\frac{p-1}{6}}+b)}{(1+a+b)}$
\end{itemize}

\end{document}