\documentclass[12pt]{article}

\usepackage{amsthm}
\usepackage{amssymb}

\newtheorem{lemma}{Lemma}
 
\begin{document}


Let $p \equiv 1 \bmod{3}$. Let $F(X)=X^{\frac{p+1}{2}}+aX^{\frac{p+5}{6}}+bX$ be a polynomial over $\mathbb{F}_{p}$. Let $\alpha$ be a primitive root in $\mathbb{F}_{p}$.
\\
Notice that $F(X)=X(X^{\frac{p-1}{2}}+aX^{\frac{p-1}{6}}+b)$. We will use the approach of considering the $\alpha^{6k+r},r=0,...,5$. This divides $F(X)$ into 6 classes:

\begin{itemize}
	\item
		$F(\alpha^{6k}) = \alpha^{6k}(1+a+b)$
	\item
		$F(\alpha^{6k+1})= \alpha^{6k+1}(-1+a\alpha^{\frac{p-1}{6}}+b)$
	\item
		$F(\alpha^{6k+2})=\alpha^{6k+2}(1+a\alpha^{\frac{p-1}{3}}+b)$
	\item
		$F(\alpha^{6k+3})=\alpha^{6k+3}(-1-a+b)$
	\item
		$F(\alpha^{6k+4})=\alpha^{6k+4}(1+a\alpha^{2\frac{p-1}{3}}+b)$
	\item
		$F(\alpha^{6k+5})=\alpha^{6k+5}(-1+a\alpha^{5\frac{p-1}{6}}+b)$
\end{itemize}

\section{Lemas for PP} % (fold)
\label{sec:lemas_for_pp}
Note that in order for $F(X)$ to be a PP, these 6 "classes" should be disjoint. This is $F(\alpha^{6k+r}) \neq F(\alpha^{6l+s})$ where $k,l < \frac{p-1}{6}$, $r,s < 6$, and $r \neq s$. We want to find the necessary conditions on $a$ and $b$ such that this occurs.

\begin{lemma}[Lemma 1:]
	Let $F(X)$ be defined as above. In order for $F(X)$ to be a Permutation Polynomial, $a$ and $b$ can NOT satisfy any of the following equalities:
\end{lemma}

\textbf{Class $6k$}

\begin{itemize}
	\item Class $6l+1$ \\ \\
		$\alpha^{6k}(1+a+b) = \alpha^{6l+1}(-1+a\alpha^{\frac{p-1}{6}}+b)$
		\\ \\
		$\alpha^{6(k-l)}(1+a+b) = \alpha(-1+a\alpha^{\frac{p-1}{6}}+b)$
		\\ \\
		$\alpha^{6(k-l)} =  \alpha \frac{(-1+a\alpha^{\frac{p-1}{6}}+b)}{(1+a+b)} $
		\\ \\
		From this we know that $a$ and $b$ must satisfy
		 $\alpha^{6m} \neq  \alpha \frac{(-1+a\alpha^{\frac{p-1}{6}}+b)}{(1+a+b)} $

	\item Class $6l+2$ \\ \\
		$\alpha^{6k}(1+a+b) = \alpha^{6l+2}(1+a\alpha^{\frac{p-1}{3}}+b)$
		\\ \\
		$\alpha^{6(k-l)}(1+a+b) = \alpha^{2}(1+a\alpha^{\frac{p-1}{3}}+b)$
		\\ \\
		$\alpha^{6(k-l)} = \alpha^{2}\frac{(1+a\alpha^{\frac{p-1}{3}}+b)}{(1+a+b)}$
		\\ \\
		From this we know that $a$ and $b$ must satisfy
		$\alpha^{6m} \neq \alpha^{2}\frac{(1+a\alpha^{\frac{p-1}{3}}+b)}{(1+a+b)}$

	\item Class $6l+3$ \\ \\
		$\alpha^{6k}(1+a+b) = \alpha^{6l+3}(-1-a+b)$
		\\ \\
		$\alpha^{6(k-l)}(1+a+b) = \alpha^{3}(-1-a+b)$
		\\ \\
		$\alpha^{6(k-l)} = \alpha^{3}\frac{(-1-a+b)}{(1+a+b)}$
		\\ \\
		From this we know that $a$ and $b$ must satisfy
		$\alpha^{6m} \neq \alpha^{3}\frac{(-1-a+b)}{(1+a+b)}$
	
	\item Class $6l+4$ \\ \\
		$\alpha^{6k}(1+a+b) = \alpha^{6l+4}(1+a\alpha^{2\frac{p-1}{3}}+b)$
		\\ \\
		$\alpha^{6(k-l)}(1+a+b) = \alpha^{4}(1+a\alpha^{2\frac{p-1}{3}}+b)$
		\\ \\
		$\alpha^{6(k-l)} = \alpha^{4}\frac{(1+a\alpha^{2\frac{p-1}{3}}+b)}{(1+a+b)}$
		\\ \\
		From this we know that $a$ and $b$ must satisfy $\alpha^{6m} \neq \alpha^{4}\frac{(1+a\alpha^{2\frac{p-1}{3}}+b)}{(1+a+b)}$

	\item Class $6l+5$ \\ \\
		$\alpha^{6k}(1+a+b) = \alpha^{6l+5}(-1+a\alpha^{5\frac{p-1}{6}}+b)$
		\\ \\
		$\alpha^{6(k-l)}(1+a+b) = \alpha^{5}(-1+a\alpha^{5\frac{p-1}{6}}+b)$
		\\ \\
		$\alpha^{6(k-l)} = \alpha^{5}\frac{(-1+a\alpha^{5\frac{p-1}{6}}+b)}{(1+a+b)}$
		\\ \\
		From this we know that $a$ and $b$ must satisfy $\alpha^{6m} \neq \alpha^{5}\frac{(-1+a\alpha^{5\frac{p-1}{6}}+b)}{(1+a+b)}$
\end{itemize}
% section lemas_for_pp (end)

\section{No zeros} % (fold)
\label{sec:no_zeros}
Recall that for any polynomial to be a permutation polynomial it can only have $1$ root. This provides another necessary condition for $F(X)$ to be PP. These $6$ classes cannot be equal to $0$. This is because $\alpha^{n} \neq 0$ for all $n$. We find necessary conditions on $a$ and $b$  by equating our partitions to $0$.

\begin{lemma}[Lemma 1:]
	Let $F(X)$ be defined as above. We get the following conditions on $a$ and $b$ by contradiction:
\end{lemma}

\begin{itemize}
	\item $F(\alpha^{6k}) = 0 \rightarrow [a,-1-a]$ is not a possible combination.
	\item $F(\alpha^{6k+1}) = 0 \rightarrow [a,1-a\alpha^{\frac{p-1}{6}}]$ is not a possible combination.
	\item $F(\alpha^{6k+2}) = 0 \rightarrow [a,-1-a\alpha^{\frac{p-1}{3}}]$ is not a possible combination.
	\item $F(\alpha^{6k+3}) = 0 \rightarrow [a,1+a]$ is not a possible combination.
	\item $F(\alpha^{6k+4}) = 0 \rightarrow [a,-1-a\alpha^{2*\frac{p-1}{3}}]$ is not a possible combination.
	\item $F(\alpha^{6k+5}) = 0 \rightarrow [a,1-a\alpha^{5*\frac{p-1}{6}}]$ is not a possible combination. 
\end{itemize}
% section no_zeros (end)

\end{document}