\documentclass[12pt]{article}

\hoffset=-0.125in

\usepackage{amsthm}
\usepackage{amssymb}
\usepackage{amsmath}

\newtheorem{conjecture}{Conjecture}
\newtheorem{example}{Example}
\newtheorem{definition}{Definition}
\newtheorem{proposition}{Proposition}
\newtheorem{lemma}{Lemma}
\newtheorem{theorem}{Theorem}

\title{Technical Report on a Class of Permutation Polynomials}

\author{Christian A. Rodr\'{\i}guez \\ Alex D. Santos \\ University of Puerto Rico \\ Rio Piedras Campus \\ Department of Computer Science}
\date{}

\begin{document}
\maketitle

\begin{abstract}

El abstract lo dejamos para el final

\end{abstract}

\section{Introduction}\label{intro}

We are studying the coefficients $a$ and $b$ that make a polynomial a permutation polynomial. Let $p \equiv 1 \bmod{3}$. We consider the polynomial $F(x)=x^{\frac{p+1}{2}} + ax^{\frac{p+5}{6}} + bx$ defined over a finite field $\mathbb{F}_{q}$. \\
\\
1. Results on binomials \\
2. Francis' motivation to chose $F(x)$ \\
\\
Recall that all elements in $\mathbb{F}_{q}$ can be expressed as a power of the primitive root $\alpha \in \mathbb{F}_{q}$. Our approach in studying $F(x)$ is to use the division algorithm to consider $x=\alpha^{n}$ where $n=6k+r$. This partitions $F(x)$ into 6 classes:

\begin{itemize}
	\item
		$F(\alpha^{6k}) = \alpha^{6k}(1+a+b)$
	\item
		$F(\alpha^{6k+1})= \alpha^{6k}(-\alpha+a\alpha^{\frac{p+5}{6}}+b\alpha)$
	\item
		$F(\alpha^{6k+2})=\alpha^{6k}(\alpha^{2}+a\alpha^{\frac{p+5}{3}}+b\alpha^{2})$
	\item
		$F(\alpha^{6k+3})=\alpha^{6k}(-\alpha^{3}-a\alpha^{3}+b\alpha^{3})$
	\item
		$F(\alpha^{6k+4})=\alpha^{6k}(\alpha^{4}+a\alpha^{2\frac{p+5}{3}}+b\alpha^{4})$
	\item
		$F(\alpha^{6k+5})=\alpha^{6k}(-\alpha^{5}+a\alpha^{5\frac{p+5}{6}}+b\alpha^{5})$ 
\end{itemize}



\section{Preliminaries}\label{prelim}

We study polynomials defined over finite fields.

\begin{definition}
	A \textbf{finite field} $\mathbb{F}_{q}$, $q=p^{r}$, $p$ prime is a field with $q$ elements.
\end{definition}

Specifically, we study polynomials that permute the elements of the field. This is, polynomials that when evaluated over the field produce all elements in the field.
\begin{definition}
	A polynomial $f(x)$ defined over $\mathbb{F}_{q}$ is called a \textbf{permutation polynomial} if $f(x)$ acts as a permutation over the elements of $\mathbb{F}_{q}$.
\end{definition}

Our approach in studying these permutation polynomials utilizes two important concepts. The first is primitive roots of finite fields.
\begin{definition}
	A \textbf{primitive root} $\alpha$ of a finite field $\mathbb{F}_{q}$ is a generator of the multiplicative group $\mathbb{F}_{q}^{\times}$
\end{definition}

The second important concept is the Division Algorithm.
\begin{theorem}[Division Algorithm]
	Given integers $a$ and $b$, with $b>0$ there exists unique integers $q$ and $r$ satisfying $a=qb+r$, $0 \leq r < b$
\end{theorem}

\section{The amount of Permutation Polynomials of our class is divisible by $2$}\label{divby2}


In our study of possible pairs $(a,b)$ that produce permutation polynomials, examples we have calculated led us to the following conjecture.
\begin{conjecture}
	Consider the polynomial $F(x)$. If $(a,b)$ produces a permutation, then $(a,-b)$ also produces a permutation.
\end{conjecture}

In the case of $q=31$ we have proved this conjecture. We found a correspondence between the classes we defined above by evaluating our polynomial in $(a,b)$ and $(a,-b)$. This way we proved that whenever one of the pairs produces a permutation polynomial, so does the other.


\begin{proof}
	Let $P_{31}(x,a,b) = x(x^{\frac{p-1}{2}}+ax^{\frac{p-1}{6}}+b)$ defined over $\mathbb{F}_{31}$. We will prove that $P_{31}(\alpha^{6k+i},a,b) = P_{31}(\alpha^{6l+j},a,-b)$ where
	$$
	l =
	\begin{cases}
	k+2 \bmod{5}, & 0 \leq i \leq 2 \\
	k+3 \bmod{5}, & 3 \leq i \leq 5
	\end{cases}
	$$
	,
	$$
	j =
	\begin{cases}
	i+3, & 0 \leq i \leq 2 \\
	i-3, & 3 \leq i \leq 5
	\end{cases}
	$$
	
	First note that
	\begin{align*}
	&P_{31}(\alpha^{6k+i},a,b) \\
	&=\alpha^{6k+i}((\alpha^{6k+i})^{\frac{p-2}{2}}+a(\alpha^{6k+i})^{\frac{p-1}{6}}+b) \\
	&=\alpha^{6k+i}((-1)^{i}+a\alpha^{i\frac{p-1}{6}}+b)
	\end{align*}

	Also note that
	\begin{align*}
	&6(k+2)+i+3=6k+12+i+3=6k+i+15 \\
	&6(k+3)+i-3=6k+18+i-3=6k+i+15
	\end{align*}

	Finally:
	\begin{align*}
	&P_{31}(\alpha^{6l+j},a,-b)	\\
	&= -\alpha^{6k+i}((-\alpha^{6k+i})^\frac{p-1}{2}+a(-\alpha^6k+i)^\frac{p-1}{6}-b) \\
	&= -\alpha^{6k+i}((-1)^{\frac{p-1}{2}}(\alpha^{6k+i})^\frac{p-1}{2}+a(-1)^{\frac{p-1}{6}}(\alpha^6k+i)^\frac{p-1}{6}-b) \\
	&= -\alpha^{6k+i}(-(-1)^{i}-a\alpha^{i\frac{p-1}{6}}-b) \\
	&= \alpha^{6k+i}((-1)^{i}+a\alpha^{i\frac{p-1}{6}}+b) \\
	\end{align*}
	
\end{proof}

Our proof utilizes the fact that $\frac{p-1}{2} = \frac{30}{2}=15$ is odd. In the generalization there must exist another variable that fixes this fact when $\frac{p-1}{2}$ is even.


\begin{thebibliography}{}

We need to add references.

\end{thebibliography}

\end{document}