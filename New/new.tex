\documentclass{article}

%\documentclass{proc-l}

\usepackage{amssymb}
\usepackage{amsmath}
\usepackage{amsfonts}
\usepackage{amsthm}



\newcommand{\C}{\mathcal{C}}
\newcommand{\sN}{\mathcal{N}}
\newcommand{\M}{\mathcal{M}}
\newcommand{\X}{\mathbf{X}}
\newcommand{\T}{\mathcal{T}}

\newcommand{\vv} {{\boldsymbol{\nu}}}
\newcommand{\x}{{\bf x}}
\newcommand{\xx}{\mathbf{x}}
\newcommand{\be}{{\bf e}}
\newcommand{\ff}{{\mathbb{ F\!}}}
\newcommand{\lff}{{\mathbb{ F\!}}^{\,\prime}}
\newcommand{\pp}{{\mathbb{P\!}}}
\newcommand{\Fq}{{\mathbb{F\!}_q}}
\newcommand{\Fp}{{\mathbb{F\!}_p}}
\newcommand{\FF}{\mathbb{F}_2}
\newcommand{\F}{\mathbb{F}}
\newcommand{\QQ}{\mathbb{Q}_2}
\newcommand{\Qp}{\mathbb{Q}_p}
\newcommand{\Q}{\mathbb{Q}}
\newcommand{\Z}{\mathbb{Z}}
\newcommand{\Zp}{\mathbb{Z}_p}
\newcommand{\K}{K}
\newcommand{\N}{\mathbb{N}}
\newcommand{\p}{$p$\nobreakdash}
%\newcommand{\e}{\mathbf{e}}
\newcommand{\tvec}{\mathbf{t}}
\newcommand{\OKxi}{\mathcal{O}_{\K(\xi)}}
\DeclareMathOperator{\tr}{Tr}
\def\Tr{\mathop{\rm Tr}\nolimits}
\newcommand{\cc}{\mathcal{C}}
\newcommand{\il}{\mathcal{I}}
\newcommand{\ee}{\epsilon}
\newcommand{\bb}{\beta}

%%%%%%%%%%%%%%%%%%%%%%%%
% AMS Proc Styles
%%%%%%%%%%%%%%%%%%%%%%

\newtheorem{theorem}{Theorem}[section]
\newtheorem{lemma}[theorem]{Lemma}

\theoremstyle{definition}
\newtheorem{definition}[theorem]{Definition}

%\theoremstyle{corollary}
\newtheorem{corollary}[theorem]{Corollary}

\newtheorem{example}[theorem]{Example}
\newtheorem{xca}[theorem]{Exercise}

\newtheorem{construction}[theorem]{Construction}


\newtheorem{prop}[theorem]{Proposition}

\theoremstyle{remark}
\newtheorem{remark}[theorem]{Remark}

\numberwithin{equation}{section}




\begin{document}

\title{On a Class of Permutation Polynomials over Finite Fields}


\maketitle

\begin{abstract}

\end{abstract}


%%%%%%%%%%%%%%%%%%%%%%%%%%%%%%%%%%%%%%%%%%%%%%%%%%%%%%%%%%%%%%%%%%%%%%%
\section{Results}
%%%%%%%%%%%%%%%%%%%%%%%%%%%%%%%%%%%%%%%%%%%%%%%%%%%%%%%%%%%%%%%%%%%%%%%

\begin{definition}

  Sea $a = \alpha^i, b = \alpha^j$ y $\sim$ la relacion definida por $(a,b) \sim (a', b')$ 
  $<=> a' = \alpha^{i+h(\frac{p-1}{d_1} - \frac{p-1}{d_2})}, b' = \alpha^{j+h(\frac{p-1}{d_1})}$

\end{definition}

\begin{prop}
  
  $\sim$ definida arriba es una relaci\'on de equivalencia.

\end{prop}

\begin{proof}
  
  1. Sea $a=\alpha^i$, $b=\alpha^j$ y escoja $h=0$. Entonces $a' = \alpha^{i+0(\frac{p-1}{d_1}-\frac{p-1}{d_2})} = \alpha^i = a$ y $b' = \alpha^{j+0(\frac{p-1}{d_1})} = \alpha^j = b$. Por lo tanto $(a,b) \sim (a,b)$ y la relacion es reflexiva.

  2. Sea $a = \alpha^i$, $b=\alpha^j$, $a' = \alpha^{i+h(\frac{p-1}{d_1}-\frac{p-1}{d_2})}$ y $b' = \alpha^{j+h(\frac{p-1}{d_1})}$ entonces $(a,b) \sim (a',b')$. Queremos encontrar $l$ tal que $a = \alpha^{i+h(\frac{p-1}{d_1}-\frac{p-1}{d_2})+l(\frac{p-1}{d_1}-\frac{p-1}{d_2})}$ y $b = \alpha^{j+h(\frac{p-1}{d_1})+l(\frac{p-1}{d_1})}$. Escoja $l=d_1d_2-h$, entonces obtenemos: $\alpha^{i+d_1d_2(\frac{p-1}{d_1}-\frac{p-1}{d_2})} = \alpha^i = a$ y $\alpha^{j+d_1d_2(\frac{p-1}{d_1})} = \alpha^j = b$. Por lo tanto $(a',b') \sim (a,b) $ y la relacion es simetrica.

  3. Suponga que $a = \alpha^i$, $b=\alpha^j$, $a' = \alpha^{i+h(\frac{p-1}{d_1}-\frac{p-1}{d_2})}$, $b' = \alpha^{j+h(\frac{p-1}{d_1})}$, $a'' = \alpha^{i+h(\frac{p-1}{d_1}-\frac{p-1}{d_2})+l(\frac{p-1}{d_1}-\frac{p-1}{d_2})}$, $b'' = \alpha^{j+h(\frac{p-1}{d_1})+l(\frac{p-1}{d_1})}$. Por lo tanto $(a,b) \sim (a',b')$ y $(a',b') \sim (a'',b'')$. Ahora note que $a'' = \alpha^{i+(h+l)(\frac{p-1}{d_1}-\frac{p-1}{d_2})}$, $b'' = \alpha^{j+(h+l)(\frac{p-1}{d_1})}$, por lo tanto $(a,b) \sim (a'',b'')$ y la relacion es transitiva.

  Como la relacion es reflexiva, simetrica y transitiva, concluimos que es una relacion de equivalencia.

\end{proof}

\begin{prop}
  
  Sea $[a, b]$ la clase de equivalencia de $(a, b)$. Si $(a', b') \in [a, b]$, entonces 
  $|V_{a', b'}| = |V_{a, b}|$

\end{prop}

\begin{proof}
  
  Sea $\alpha$ la raiz primitiva del cuerpo finito. $$F_{a', b'}(\alpha^{k+1}) = \alpha^{k+1}((\alpha^{k+1})^{\frac{p-1}{d_1}} + \alpha^{i + \frac{p-1}{d_1} - \frac{p-1}{d_2}}(\alpha^{k+1})^{\frac{p-1}{d_2}} + \alpha^{j + \frac{p-1}{d_1}})$$

  $$= \alpha^{k+1}((\alpha^{k})^{\frac{p-1}{d_1}} \cdot \alpha^{\frac{p-1}{d_1}} + \alpha^{i} \cdot \frac{\alpha^{\frac{p-1}{d_1}}} {\alpha^{\frac{p-1}{d_2}}} (\alpha^{k})^{\frac{p-1}{d_2}} \cdot \alpha^{\frac{p-1}{d_2}} + \alpha^{j} \cdot \alpha^{\frac{p-1}{d_1}})$$

  $$= \alpha^{\frac{p-1}{d_1} + 1} \cdot \alpha^{k}((\alpha^{k})^{\frac{p-1}{d_1}} + \alpha^{i}(\alpha^{k})^{\frac{p-1}{d_2}} + \alpha^{j} )$$

  $$= C \cdot F_{a,b}(\alpha^k), \mbox{donde } C = \alpha^{\frac{p-1}{d_1} + 1}$$

  En general para cada termino de $F_{a,b}(\alpha^{k})$ va a haber un termino correspondiente de $F_{a',b'}(\alpha^{k+1})$ donde $a' = \alpha^{i + h(\frac{p-1}{d_1} - \frac{p-1}{d_2})}$ y $b'= \alpha^{j + h(\frac{p-1}{d_1})}$. Por otra parte, debe ser el caso de que $\left\vert V_{F_{a,b}} \right\vert = \left\vert V_{F_{a',b'}} \right\vert$.

  Sea $f:V_{a', b'} \rightarrow \alpha^{\frac{p-1}{d_1}}V_{a, b}$ dada por $f(F_{a', b'}(\alpha^{k+1})) = \alpha^{\frac{p-1}{d_1}+1}F_{a, b}(\alpha^k)$.
  Suponga que $f(F_{a', b'}(\alpha^{k_1+1})) = f(F_{a', b'}(\alpha^{k_2+1}))$ donde $k_1, k_2 \in \mathbb{F}_q$. 

  Considere $f(F_{a', b'}(\alpha^{k_1+1}))$

  $$= f(\alpha^{k_1+1}((\alpha^{k_1+1})^{\frac{p-1}{d_1}} + \alpha^{i}(\alpha^{k_1+1})^{\frac{p-1}{d_2}} + \alpha^{j}))$$ 

  $$= \alpha^{\frac{p-1}{d_1}+1}(\alpha^{k_1}((\alpha^{k_1})^{\frac{p-1}{d_1}} + \alpha^{i}(\alpha^{k_1})^{\frac{p-1}{d_2}} + \alpha^{j}))$$ 

  $$= \alpha^{k_1+1}(\alpha^{\frac{p-1}{d_1}}((\alpha^{k_1})^{\frac{p-1}{d_1}} + \alpha^{i}(\alpha^{k_1})^{\frac{p-1}{d_2}} + \alpha^{j}))$$ 

  $$= \alpha^{k_1+1}((\alpha^{k_1+1})^{\frac{p-1}{d_1}} + \alpha^{i + \frac{p-1}{d_1} - \frac{p-1}{d_2} + \frac{p-1}{d_2}}(\alpha^{k_1})^{\frac{p-1}{d_2}} + \alpha^{j + \frac{p-1}{d_1}})$$

  $$= \alpha^{k_1+1}((\alpha^{k_1+1})^{\frac{p-1}{d_1}} + \alpha^{i + \frac{p-1}{d_1} - \frac{p-1}{d_2}}(\alpha^{k_1+1})^{\frac{p-1}{d_2}} + \alpha^{j + \frac{p-1}{d_1}})$$ 

  $$= F_{a', b'}(\alpha^{k_1+1})$$

  Luego considere $f(F_{a', b'}(\alpha^{k_2+1}))$

  $$= f(\alpha^{k_2+1}((\alpha^{k_2+1})^{\frac{p-1}{d_1}} + \alpha^{i}(\alpha^{k_2+1})^{\frac{p-1}{d_2}} + \alpha^{j}))$$

  $$ = \alpha^{\frac{p-1}{d_1}+1}(\alpha^{k_2}((\alpha^{k_2})^{\frac{p-1}{d_1}} + \alpha^{i}(\alpha^{k_2})^{\frac{p-1}{d_2}} + \alpha^{j}))$$

  $$ = \alpha^{k_2+1}(\alpha^{\frac{p-1}{d_1}}((\alpha^{k_2})^{\frac{p-1}{d_1}} + \alpha^{i}(\alpha^{k_2})^{\frac{p-1}{d_2}} + \alpha^{j}))$$

  $$ = \alpha^{k_2+1}((\alpha^{k_2 + 1})^{\frac{p-1}{d_1}} + \alpha^{i + \frac{p-1}{d_1} - \frac{p-1}{d_2} + \frac{p-1}{d_2}}(\alpha^{k_2})^{\frac{p-1}{d_2}} + \alpha^{j + \frac{p-1}{d_1}})$$

  $$= \alpha^{k_2+1}((\alpha^{k_2 + 1})^{\frac{p-1}{d_1}} + \alpha^{i + \frac{p-1}{d_1} - \frac{p-1}{d_2}}(\alpha^{k_2 + 1})^{\frac{p-1}{d_2}} + \alpha^{j + \frac{p-1}{d_1}})$$

  $$= F_{a', b'}(\alpha^{k_2+1})$$

  En conclusi\'on $F_{a', b'}(\alpha^{k_1+1}) = F_{a', b'}(\alpha^{k_2+1})$ por lo tanto $f$ es una funci\'on $1-1$

  Considere un elemento en el campo de valores dado por $\alpha^{\frac{p-1}{d_1}}F_{a, b}(\alpha^k)$

  $$\alpha^{\frac{p-1}{d_1}}F_{a, b}(\alpha^k) = \alpha^{\frac{p-1}{d_1}+1}(\alpha^{k}((\alpha^{k})^{\frac{p-1}{d_1}} + \alpha^{i}(\alpha^{k})^{\frac{p-1}{d_2}} + \alpha^{j}))$$

  $$ = \alpha^{k+1}(\alpha^{\frac{p-1}{d_1}}((\alpha^{k})^{\frac{p-1}{d_1}} + \alpha^{i}(\alpha^{k})^{\frac{p-1}{d_2}} + \alpha^{j}))$$

  $$ = \alpha^{k+1}((\alpha^{k + 1})^{\frac{p-1}{d_1}} + \alpha^{i + \frac{p-1}{d_1} - \frac{p-1}{d_2} + \frac{p-1}{d_2}}(\alpha^{k})^{\frac{p-1}{d_2}} + \alpha^{j + \frac{p-1}{d_1}})$$

  $$= \alpha^{k+1}((\alpha^{k + 1})^{\frac{p-1}{d_1}} + \alpha^{i + \frac{p-1}{d_1} - \frac{p-1}{d_2}}(\alpha^{k + 1})^{\frac{p-1}{d_2}} + \alpha^{j + \frac{p-1}{d_1}})$$

  $$= F_{a', b'}(\alpha^{k+1})$$

  En conclusi\'on para cada elemento en el campo de valores, $\alpha^{\frac{p-1}{d_1}}F_{a, b}(\alpha^k)$, existe un elemento en el dominio, $F_{a', b'}(\alpha^{k+1})$. Por lo tanto $f$ es una funci\'on sobre.

\end{proof}

\begin{prop}
  
  $|[a, b]| = lcm(d_1,d_2)$

\end{prop}

\begin{proof}

  Suponga que $a=\alpha^i$, $b=\alpha^j$. Note que podemos obtener los elementos de $[a,b]$ aplicando la transformacion $f:(a,b) \rightarrow ( a\cdot\alpha^{(\frac{p-1}{d_1} - \frac{p-1}{d_2})}, b\cdot\alpha^{(\frac{p-1}{d_1})} )$ multiples veces. Ahora note que:

  \begin{align*}
  & f(a\cdot \alpha^{i+(lcm(d_1,d_2) - 1)(\frac{p-1}{d_1} - \frac{p-1}{d_2})}, b\cdot\alpha^{j+(lcm(d_1,d_2) - 1)(\frac{p-1}{d_1})}) \\
  & = (\alpha^{i+lcm(d_1,d_2)(\frac{p-1}{d_1}-\frac{p-1}{d_2})}, \alpha^{j+lcm(d_1,d_2)(\frac{p-1}{d_1})}) \\
  & = (\alpha^{i+lcm(d_1,d_2)(\frac{p-1}{d_1})-lcm(d_1,d_2)(\frac{p-1}{d_2})}, \alpha^{j+lcm(d_1,d_2)(\frac{p-1}{d_1})}) \\
    & = (\alpha^{i+\frac{d_1d_2}{gcd(d_1,d_2)}(\frac{p-1}{d_1})-\frac{d_1d_2}{gcd(d_1,d_2)}(\frac{p-1}{d_2})}, \alpha^{j+\frac{d_1d_2}{gcd(d_1,d_2)}(\frac{p-1}{d_1})}) \\
  & = (\alpha^{i+\frac{d_2}{gcd(d_1,d_2)}(p-1)-\frac{d_2}{gcd(d_1,d_2)}(p-1)}, \alpha^{j+\frac{d_2}{gcd(d_1,d_2)}(p-1)}) \\
  & = (\alpha^i, \alpha^j)
  \end{align*}

  Por lo tanto al aplicar la transformacion $lcm(d_1,d_2)$ veces, tendremos una cadena de elementos en $[a,b]$. Ahora suponga que existe $c < lcm(d_1,d_2)$ tal que $\alpha^{i+c(\frac{p-1}{d_1} - \frac{p-1}{d_2})} = \alpha^i$ y $\alpha^{j+c(\frac{p-1}{d_1})} = \alpha^j$. Esto implica que $\alpha^{c(\frac{p-1}{d_1} - \frac{p-1}{d_2})} = 1$, luego $\alpha^{c(\frac{p-1}{d_1}) - c(\frac{p-1}{d_2})} = 1$, esto solo es posible si $c$ es multiplo de $d_1$ y $d_2$ pero $c < lcm(d_1,d_2)$ y $lcm(d_1,d_2)$ es el elemento mas pequeno tal que esto ocurre. Por lo tanto la cantidad de elementos en la clase de equivalencia $[ a,b]$ es de tama\~no $lcm(d_1,d_2)$.
  
\end{proof}

\begin{prop}
  
  Suponga que $d_2 = d_1 \cdot h + r$, $1 \leq r \geq d_1$. Entonces, $|[a, b]| = 
  \frac{d_1 \cdot d_2}{?}$

\end{prop}

\begin{prop}

  El n\'umero de polinomios $F_{a', b'}(x)$ con $|V_{a, b}|$ es un m\'ultiplo de $|[a, b]|$

\end{prop}

 
  
\end{document}

