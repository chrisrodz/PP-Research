\documentclass{article}

%\documentclass{proc-l}

\usepackage{amssymb}
\usepackage{amsmath}
\usepackage{amsfonts}
\usepackage{amsthm}



\newcommand{\C}{\mathcal{C}}
\newcommand{\sN}{\mathcal{N}}
\newcommand{\M}{\mathcal{M}}
\newcommand{\X}{\mathbf{X}}
\newcommand{\T}{\mathcal{T}}

\newcommand{\vv} {{\boldsymbol{\nu}}}
\newcommand{\x}{{\bf x}}
\newcommand{\xx}{\mathbf{x}}
\newcommand{\be}{{\bf e}}
\newcommand{\ff}{{\mathbb{ F\!}}}
\newcommand{\lff}{{\mathbb{ F\!}}^{\,\prime}}
\newcommand{\pp}{{\mathbb{P\!}}}
\newcommand{\Fq}{{\mathbb{F\!}_q}}
\newcommand{\Fp}{{\mathbb{F\!}_p}}
\newcommand{\FF}{\mathbb{F}_2}
\newcommand{\F}{\mathbb{F}}
\newcommand{\QQ}{\mathbb{Q}_2}
\newcommand{\Qp}{\mathbb{Q}_p}
\newcommand{\Q}{\mathbb{Q}}
\newcommand{\Z}{\mathbb{Z}}
\newcommand{\Zp}{\mathbb{Z}_p}
\newcommand{\K}{K}
\newcommand{\N}{\mathbb{N}}
\newcommand{\p}{$p$\nobreakdash}
%\newcommand{\e}{\mathbf{e}}
\newcommand{\tvec}{\mathbf{t}}
\newcommand{\OKxi}{\mathcal{O}_{\K(\xi)}}
\DeclareMathOperator{\tr}{Tr}
\def\Tr{\mathop{\rm Tr}\nolimits}
\newcommand{\cc}{\mathcal{C}}
\newcommand{\il}{\mathcal{I}}
\newcommand{\ee}{\epsilon}
\newcommand{\bb}{\beta}

%%%%%%%%%%%%%%%%%%%%%%%%
% AMS Proc Styles
%%%%%%%%%%%%%%%%%%%%%%

\newtheorem{theorem}{Theorem}[section]
\newtheorem{lemma}[theorem]{Lemma}

\theoremstyle{definition}
\newtheorem{definition}[theorem]{Definition}

%\theoremstyle{corollary}
\newtheorem{corollary}[theorem]{Corollary}

\newtheorem{example}[theorem]{Example}
\newtheorem{xca}[theorem]{Exercise}

\newtheorem{construction}[theorem]{Construction}


\newtheorem{prop}[theorem]{Proposition}

\theoremstyle{remark}
\newtheorem{remark}[theorem]{Remark}

\numberwithin{equation}{section}




\begin{document}

\title{On a Class of Permutation Polynomials over Finite Fields}


\maketitle

\begin{abstract}

\end{abstract}


%%%%%%%%%%%%%%%%%%%%%%%%%%%%%%%%%%%%%%%%%%%%%%%%%%%%%%%%%%%%%%%%%%%%%%%
\section{Results}
%%%%%%%%%%%%%%%%%%%%%%%%%%%%%%%%%%%%%%%%%%%%%%%%%%%%%%%%%%%%%%%%%%%%%%%

\begin{definition}

  Sea $a = \alpha^i, b = \alpha^j$ y $\sim$ la relacion definida por $(a,b) \sim (a', b')$ 
  $<=> a' = \alpha^{i+h(\frac{p-1}{d_1} - \frac{p-1}{d_2})}, b' = \alpha^{j+h(\frac{p-1}{d_1})}$

\end{definition}

\begin{prop}
  
  $\sim$ definida arriba es una relaci\'on de equivalencia.

\end{prop}

\begin{proof}
  
  Pendiente

\end{proof}

\begin{prop}
  
  Sea $[a, b]$ la clase de equivalencia de $(a, b)$. Si $(a', b') \in [a, b]$, entonces 
  $|V_{a', b'}| = |V_{a, b}|$

\end{prop}

\begin{proof}
  
  Sea $\alpha$ la raiz primitiva del cuerpo finito. $$F_{a', b'}(\alpha^{k+1}) = \alpha^{k+1}((\alpha^{k+1})^{\frac{p-1}{d_1}} + \alpha^{i + \frac{p-1}{d_1} - \frac{p-1}{d_2}}(\alpha^{k+1})^{\frac{p-1}{d_2}} + \alpha^{j + \frac{p-1}{d_1}})$$

  $$= \alpha^{k+1}((\alpha^{k})^{\frac{p-1}{d_1}} \cdot \alpha^{\frac{p-1}{d_1}} + \alpha^{i} \cdot \frac{\alpha^{\frac{p-1}{d_1}}} {\alpha^{\frac{p-1}{d_2}}} (\alpha^{k})^{\frac{p-1}{d_2}} \cdot \alpha^{\frac{p-1}{d_2}} + \alpha^{j} \cdot \alpha^{\frac{p-1}{d_1}})$$

  $$= \alpha^{\frac{p-1}{d_1} + 1} \cdot \alpha^{k}((\alpha^{k})^{\frac{p-1}{d_1}} + \alpha^{i}(\alpha^{k})^{\frac{p-1}{d_2}} + \alpha^{j} )$$

  $$= C \cdot F_{a,b}(\alpha^k), \mbox{donde } C = \alpha^{\frac{p-1}{d_1} + 1}$$

  En general para cada termino de $F_{a,b}(\alpha^{k})$ va a haber un termino correspondiente de $F_{a',b'}(\alpha^{k+1})$ donde $a' = \alpha^{i + h(\frac{p-1}{d_1} - \frac{p-1}{d_2})}$ y $b'= \alpha^{j + h(\frac{p-1}{d_1})}$. Por otra parte, debe ser el caso de que $\left\vert V_{F_{a,b}} \right\vert = \left\vert V_{F_{a',b'}} \right\vert$.

\end{proof}

\begin{prop}
  
  Si $d_2 = d_1 \cdot h$, entonces $|[a, b]| = d_2$

\end{prop}

\begin{proof}
  Note that we can repeat this process using $a'' = a' \cdot \alpha^{(d+2)(\frac{q-1}{2d})}$, $b'' = b' \cdot \alpha^{(\frac{q-1}{2})}$. We argue that this process can be repeated at most $d-1$ times when $d$ is even, and $2d-1$ times when $d$ is odd.
\end{proof}

\begin{prop}
  
  Suponga que $d_2 = d_1 \cdot h + r$, $1 \leq r \geq d_1$. Entonces, $|[a, b]| = 
  \frac{d_1 \cdot d_2}{?}$

\end{prop}

\begin{prop}

  El n\'umero de polinomios $F_{a', b'}(x)$ con $|V_{a, b}|$ es un m\'ultiplo de $|[a, b]|$

\end{prop}

 
  
\end{document}

