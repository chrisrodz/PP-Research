\documentclass[12pt]{article}

\hoffset=-0.125in

\usepackage{amsthm}
\usepackage{amssymb}
\usepackage{amsmath}

\newtheorem{conjecture}{Conjecture}
\newtheorem{example}{Example}
\newtheorem{definition}{Definition}
\newtheorem{proposition}{Proposition}
\newtheorem{lemma}{Lemma}
\newtheorem{theorem}{Theorem}

\title{On a Class of Permutation Polynomials}

\author{Christian A. Rodr\'{\i}guez \\ Alex D. Santos \\ University of Puerto Rico \\ Rio Piedras Campus \\ Department of Computer Science}
\date{}

\begin{document}
\maketitle

\begin{abstract}

A polynomial $f(x)$ defined over a set $A$ is called a \textbf{permutation polynomial} if $f(x)$ acts as a permutation over the elements of $A$. This is, if $f: A \rightarrow A$ is 1-1 and onto. We studyi the coefficients $a$ and $b$ that make polynomials of the form $F_{a,b}(x)=x^{\frac{p+1}{2}} + ax^{\frac{p+5}{6}} + bx$ a permutation polynomial, $a,b \in \mathbb{F}_{q}^{\times}$. We show that this family of polynomials is rich in permutations, and that the amount of permutation polynomials for any $q$ is divisible by $6$. Our approach in studying $F(x)$ is to use the division algorithm to consider $x=\alpha^{n}$ where $n=6k+r, r=0,...,5$. If $F_{a,b}(x)$ is a permutation, this partitions $\mathbb{F}_{q}^{\times}$ into 6 classes: $F_{a,b}(\alpha^{6k+r})$ for $r=0,...,5$.

\end{abstract}

\end{document}
