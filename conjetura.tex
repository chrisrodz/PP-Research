\documentclass[12pt]{article}

\hoffset=-0.125in

\usepackage{amsthm}
\usepackage{amssymb}
\usepackage{amsmath}

\newtheorem{conjecture}{Conjecture}
\newtheorem{example}{Example}
\newtheorem{definition}{Definition}
\newtheorem{proposition}{Proposition}
\newtheorem{lemma}{Lemma}
\newtheorem{theorem}{Theorem}

\title{On a Class of Permutation Polynomials over Finite Fields}

\author{Francis Castro \\ Jos\'e Ortiz \\ Christian A. Rodr\'{\i}guez \\ Ivelisse Rubio \\ Alex D. Santos \\ University of Puerto Rico, R\'{\i}o Piedras \\ Department of Computer Science}
\date{}

\begin{document}
\maketitle

Alex y Yo demostramos la conjetura, aunque esta medio chapusiao por ahora. Lo importante es que tenemos la idea:
\begin{theorem}
Sea $p$ un primo y $d = 2\cdot m$, $m$ primo impar tal que $d\mid p-1$. Sea $F_{\alpha^i,\alpha^j}(x)$ polinomio de permutación en $\mathbb{F}_p$. Tenemos que $(\alpha^{\frac{p-1}{2}-\left(m+2\right)}\cdot F_{\alpha^i,\alpha^j}(\alpha^l)) = F_{\alpha^{i+\frac{p-1}{m}},\alpha^{j+\frac{p-1}{2}}}(\alpha^{k})$ donde $k,l \in \mathbb{F}_p^\times$ y $l=k+m+2$    
\end{theorem}


\textbf{Tenemos el problema que $l$ puede ser $0$ y eso no es bueno.(Alex sabe porque)}
\linebreak

\begin{proof}
Vamos a establecer una correspondencia entre los dos terminos. La correspondencia es la siguiente:

$F_{\alpha^{i+\frac{p-1}{m}},\alpha^{j+\frac{p-1}{2}}}(\alpha^{k}) = \alpha^k((\alpha^k)^{\frac{p-1}{2}}+(\alpha^{i+\frac{p-1}{m}})\cdot(\alpha^k)^{\frac{p-1}{d}}+\alpha^{j+\frac{p-1}{2}})$

$ = \alpha^k((\alpha^k)^{\frac{p-1}{2}}+(\alpha^{i+\frac{p-1}{m}})\cdot(\alpha^k)^{\frac{p-1}{d}}-\alpha^{j})$

$ = -\alpha^k(-(-1)^k+\alpha^{i}\cdot (\alpha^{\frac{p-1}{2}}\cdot \alpha^{\frac{p-1}{m}})\cdot(\alpha^k)^{\frac{p-1}{d}}+\alpha^{j})$

$ = -\alpha^k(-(-1)^k+\alpha^{i}\cdot (\alpha^{\frac{p-1}{2}}\cdot \alpha^{\frac{p-1}{m}} \cdot \alpha^{\frac{pk-k}{d}})+\alpha^{j})$

Note que si encontramos $l$ tal que $(\alpha^{\frac{p-1}{2}}\cdot \alpha^{\frac{p-1}{m}} \cdot \alpha^{\frac{pk-k}{d}}) = (\alpha^l)^{\frac{p-1}{d}}$ estaremos un paso mas cerca. Simplificando obtenemos

$(\alpha^{\frac{p-1}{2}}\cdot \alpha^{\frac{p-1}{m}}) = \alpha^{\frac{pd-d+pkm-mk}{md}+\frac{p-1}{2}}=\alpha^{\frac{2pd-2d+2pkm-2mk+pmd-md}{2dm}} = \alpha^{\frac{\left( 2d+2km+md \right)p-\left( 2d+2km+md \right)}{2dm}}$

$ = (\alpha^{\frac{2d+2km+md}{2m}})^{\frac{p-1}{d}} $

Por lo tanto tenemos que $l=\frac{2d+2km+md}{2m}=2+k+m$

Ahora $F_{\alpha^{i+\frac{p-1}{m}},\alpha^{j+\frac{p-1}{2}}}(\alpha^{k}) = -\alpha^k(-(-1)^k+\alpha^{i}\cdot (\alpha^{l})^{\frac{p-1}{d}}+\alpha^{j})$

Luego, $k = l-m-2$

$ = -\alpha^{l-m-2}(-(-1)^{l-m-2}+\alpha^{i}\cdot (\alpha^{l})^{\frac{p-1}{d}}+\alpha^{j})$

$ = -\alpha^{-m-2}\cdot \alpha^{l}((-1)^{l}+\alpha^{i}\cdot (\alpha^{l})^{\frac{p-1}{d}}+\alpha^{j})$

$ = (-\alpha^{-m-2}) \cdot F_{\alpha^i,\alpha^j}(\alpha^l)$
\end{proof}

Tambien demostramos la conjetura cuando $m$ es par. Ahora si que somos los duros.

\begin{theorem}
Sea $p$ un primo y $d = 2\cdot m$, $m$ primo par tal que $d\mid p-1$. Sea $F_{\alpha^i,\alpha^j}(x)$ polinomio de permutación en $\mathbb{F}_p$. Tenemos que $(-\alpha^{-m-1}\cdot F_{\alpha^i,\alpha^j}(\alpha^l)) = F_{\alpha^{i+\frac{p-1}{d}},\alpha^{j+\frac{p-1}{2}}}(\alpha^{k})$ donde $k,l \in \mathbb{F}_p^\times$ y $l=k+m+1$    
\end{theorem}

\begin{proof}
Vamos a establecer una correspondencia entre los dos terminos. La correspondencia es la siguiente:

$F_{\alpha^{i+\frac{p-1}{d}},\alpha^{j+\frac{p-1}{2}}}(\alpha^{k}) = \alpha^k((\alpha^k)^{\frac{p-1}{2}}+(\alpha^{i+\frac{p-1}{d}})\cdot(\alpha^k)^{\frac{p-1}{d}}+\alpha^{j+\frac{p-1}{2}})$

$ = \alpha^k((\alpha^k)^{\frac{p-1}{2}}+(\alpha^{i+\frac{p-1}{d}})\cdot(\alpha^k)^{\frac{p-1}{d}}-\alpha^{j})$

$ = -\alpha^k(-(-1)^k+\alpha^{i}\cdot (\alpha^{\frac{p-1}{2}}\cdot \alpha^{\frac{p-1}{d}})\cdot(\alpha^k)^{\frac{p-1}{d}}+\alpha^{j})$

$ = -\alpha^k(-(-1)^k+\alpha^{i}\cdot (\alpha^{\frac{p-1}{2}}\cdot \alpha^{\frac{p-1}{d}} \cdot \alpha^{\frac{pk-k}{d}})+\alpha^{j})$

Note que si encontramos $l$ tal que $(\alpha^{\frac{p-1}{2}}\cdot \alpha^{\frac{p-1}{d}} \cdot \alpha^{\frac{pk-k}{d}}) = (\alpha^{l})^{\frac{p-1}{d}}$ estaremos un paso mas cerca. Simplificando obtenemos

$(\alpha^{\frac{p-1}{2}}\cdot \alpha^{\frac{p-1}{m}} \cdot \alpha^{\frac{pk-k}{d}}) = \alpha^{\frac{p+pk-k-1}{d}+\frac{p-1}{2}}=\alpha^{\frac{pd-d+2p+2kp-2k-2}{2d}} = \alpha^{\frac{\left( d+2+2k \right)p-\left( d+2+2k \right)}{2d}}$

$ = (\alpha^{\frac{d+2+2k}{2}})^{\frac{p-1}{d}} $

Por lo tanto tenemos que $l=\frac{d+2+2k}{2}=k+m+1$

Ahora $F_{\alpha^{i+\frac{p-1}{m}},\alpha^{j+\frac{p-1}{2}}}(\alpha^{k}) = -\alpha^k(-(-1)^k+\alpha^{i}\cdot (\alpha^{l})^{\frac{p-1}{d}}+\alpha^{j})$

Luego, $k = l-m-1$

$ = -\alpha^{l-m-1}(-(-1)^{l-m-1}+\alpha^{i}\cdot (\alpha^{l})^{\frac{p-1}{d}}+\alpha^{j})$

$ = -\alpha^{-m-1}\cdot \alpha^{l}((-1)^{l}+\alpha^{i}\cdot (\alpha^{l})^{\frac{p-1}{d}}+\alpha^{j})$

$ = \left(-\alpha^{-m-1}\right)\cdot F_{\alpha^i,\alpha^j}(\alpha^l)$
\end{proof}

\end{document}
