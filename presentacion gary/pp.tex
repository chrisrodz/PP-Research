

\documentclass{beamer}



\mode<presentation>
{
  \usetheme{Warsaw}


}


\usepackage[english]{babel}
% or whatever

\usepackage[utf8]{inputenc}
% or whatever

\usepackage{times}
\usepackage[T1]{fontenc}
% Or whatever. Note that the encoding and the font should match. If T1
% does not look nice, try deleting the line with the fontenc.

\usepackage{amsthm}
\usepackage{amssymb}
\usepackage{amsmath}



\newtheorem{proposition}{Proposition}
\newtheorem{conjecture}{Conjecture}
\newtheorem{ourproblem}{Our Problem}


\title
{A class of Permutation Polynomials}

\author
{Christian Rodriguez\\
Alex D. Santos}

\institute[]
{
  Department of Computer Science\\
  University of Puerto Rico, Rio Piedras
}

\date
{\today}



% If you wish to uncover everything in a step-wise fashion, uncomment
% the following command: 

%\beamerdefaultoverlayspecification{<+->}

\begin{document}

\begin{frame}
  \titlepage
\end{frame}

\section{Introduction}

\subsection{The polynomial}

\begin{frame}{The polynomial}


$$F_{a,b}(x)=x^{\frac{p+1}{2}} + ax^{\frac{p+5}{6}} + bx$$


\begin{ourproblem}
Let $p \equiv 1 \bmod 3$. Find $a$ and $b$ such that $F_{a,b}(x)=x^{\frac{p+1}{2}} + ax^{\frac{p+5}{6}} + bx$ is a permutation polynomial over $\mathbb{F}_{p}$.
\end{ourproblem}


\end{frame}

\begin{frame}{The polynomial}

Let $N_p$ be the number of permutation polynomials of the type $F_{a,b}(x)=x^{\frac{p-1}{2}+1}+ax^{\frac{p-1}{6}+1}+bx$ of $\mathbb{F}_{p}$.
\begin{center}
\begin{tabular}{|c|c|}\hline
  $p$ & $N_p$ \\ \hline
  13 & 18 \\ \hline
  19& 0 \\ \hline
  31 & 18  \\ \hline
  37 & 12 \\ \hline
  43 & 36 \\ \hline
  61 & 30 \\ \hline
  67 & 108 \\ \hline
  73 & 54 \\ \hline
  79 & 48 \\ \hline
  97 & 102 \\ \hline
  103 & 72 \\  \hline
  109 & 120 \\ \hline
  \end{tabular}
\begin{tabular}{|c|c|}\hline
  $p$ & $N_p$ \\ \hline
  127 & 234 \\ \hline
  139 & 270 \\ \hline
  151 &  276 \\ \hline
  157 & 438\\ \hline
  163 & 378 \\ \hline
  181 & 552 \\ \hline
  193 & 612 \\ \hline
  199 & 624 \\ \hline
  211 & 756 \\ \hline
  223 & 540 \\ \hline
  229 & 858 \\ \hline
  241 & 828 \\ \hline
  \end{tabular}
  \end{center}

\end{frame}

\begin{frame}{The polynomial}

\begin{conjecture}
Let $p \equiv 1 \bmod 3$. The number of Permutation Polynomials over $\mathbb{F}_{p}$ of the form $F_{a,b}(x)=x^{\frac{p+1}{2}} + ax^{\frac{p+5}{6}} + bx$ is divisible by $6$.
\end{conjecture}

\end{frame}


\begin{frame}{Results}

In the case $p=31$ we have: $F_{a,b}(x)=x^{16} + ax^{6} + bx$. For the following $[a,b]$ $F(x)$ is a permutation polynomial:
$$[2, 7], [2, 24], [10, 7] [10, 24], [16,13], [16, 18],$$
$$[17, 5], [17, 26], [18, 13], [18, 18], [19, 7], [19, 24],$$
$$[22, 5], [22, 26], [23, 5], [23, 26], [28, 13], [28, 18]$$

In the case $p=37$ we have: $F_{a,b}(x)=x^{16} + ax^{6} + bx$. For the following $[a,b]$ $F(x)$ is a permutation polynomial:

$$[11, 5], [11, 32], [18, 17] [18, 20], [24,17], [24, 20],$$
$$[27, 5], [27, 32], [32, 17], [32, 20], [36, 5], [36, 32],$$


\end{frame}

\begin{frame}{Conjectures}

\begin{conjecture}
	Consider the polynomial $F(x)$. If $(a,b)$ produces a permutation, then $(a,-b)$ also produces a permutation.
\end{conjecture}

\begin{conjecture}
	The number of Permutation Polynomials over $\mathbb{F}_{p}$ of the form $F_{a,b}(x)=x^{\frac{p+1}{2}} + ax^{\frac{p+5}{6}} + bx$ is divisible by $3$.
\end{conjecture}

\end{frame}

\begin{frame}{Approach}
Our approach in studying $F(x)$ is to use the division algorithm to consider $x=\alpha^{n}$ where $n=6k+r, r=0,...,5$.
\linebreak
\linebreak
We expect that if $F_{a,b}(x)$ is a permutation, this partitions $\mathbb{F}_{q}^{\times}$ into 6 classes: $F_{a,b}(\alpha^{6k+r})$ for $r=0,...,5$
\end{frame}

\begin{frame}{Results}


  \begin{definition}
	$A_{i}=\lbrace F_{a,b}(\alpha^{6k+i}) \mid k=0,...,\frac{p-1}{6}\rbrace$
\end{definition}

\begin{lemma}
	For $i=1,...,5$ $\left\vert A_{i} \right\vert = \frac{p-1}{6}$
\end{lemma}

\end{frame}

\end{document}


