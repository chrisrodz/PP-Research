\documentclass[landscape,final,paperwidth=48in,paperheight=38in]{baposter}

\usepackage{calc}
\usepackage{graphicx}
\usepackage{amsmath}
\usepackage{amssymb}
\usepackage{relsize}
\usepackage{multirow}
\usepackage{rotating}
\usepackage{bm}
\usepackage{url} 

\usepackage{amsthm} 
\usepackage{amssymb}
\usepackage{amsmath}

\newtheorem{corollary}{Corollary}
\newtheorem{conjecture}{Conjecture}
\newtheorem{example}{Example}
\newtheorem*{definition*}{Definition}
\newtheorem{definition}{Definition}
\newtheorem{proposition}{Proposition}
\newtheorem{lemma}{Lemma}
\newtheorem{theorem}{Theorem}

\usepackage{graphicx}
\usepackage{multicol}

%\usepackage{times}
%\usepackage{helvet}
%\usepackage{bookman}
\usepackage{palatino}

\newcommand{\captionfont}{\footnotesize}

\graphicspath{{images/}{../images/}}
\usetikzlibrary{calc}

\newcommand{\SET}[1]  {\ensuremath{\mathcal{#1}}}
\newcommand{\MAT}[1]  {\ensuremath{\boldsymbol{#1}}}
\newcommand{\VEC}[1]  {\ensuremath{\boldsymbol{#1}}}
\newcommand{\Video}{\SET{V}}
\newcommand{\video}{\VEC{f}}
\newcommand{\track}{x}
\newcommand{\Track}{\SET T}
\newcommand{\LMs}{\SET L}
\newcommand{\lm}{l}
\newcommand{\PosE}{\SET P}
\newcommand{\posE}{\VEC p}
\newcommand{\negE}{\VEC n}
\newcommand{\NegE}{\SET N}
\newcommand{\Occluded}{\SET O}
\newcommand{\occluded}{o}

%%%%%%%%%%%%%%%%%%%%%%%%%%%%%%%%%%%%%%%%%%%%%%%%%%%%%%%%%%%%%%%%%%%%%%%%%%%%%%%%
%%%% Some math symbols used in the text
%%%%%%%%%%%%%%%%%%%%%%%%%%%%%%%%%%%%%%%%%%%%%%%%%%%%%%%%%%%%%%%%%%%%%%%%%%%%%%%%

%%%%%%%%%%%%%%%%%%%%%%%%%%%%%%%%%%%%%%%%%%%%%%%%%%%%%%%%%%%%%%%%%%%%%%%%%%%%%%%%
% Multicol Settings
%%%%%%%%%%%%%%%%%%%%%%%%%%%%%%%%%%%%%%%%%%%%%%%%%%%%%%%%%%%%%%%%%%%%%%%%%%%%%%%%
\setlength{\columnsep}{1.5em}
\setlength{\columnseprule}{0mm}

%%%%%%%%%%%%%%%%%%%%%%%%%%%%%%%%%%%%%%%%%%%%%%%%%%%%%%%%%%%%%%%%%%%%%%%%%%%%%%%%
% Save space in lists. Use this after the opening of the list
%%%%%%%%%%%%%%%%%%%%%%%%%%%%%%%%%%%%%%%%%%%%%%%%%%%%%%%%%%%%%%%%%%%%%%%%%%%%%%%%
\newcommand{\compresslist}{%
\setlength{\itemsep}{1pt}%
\setlength{\parskip}{0pt}%
\setlength{\parsep}{0pt}%
}

%%%%%%%%%%%%%%%%%%%%%%%%%%%%%%%%%%%%%%%%%%%%%%%%%%%%%%%%%%%%%%%%%%%%%%%%%%%%%%
%%% Begin of Document
%%%%%%%%%%%%%%%%%%%%%%%%%%%%%%%%%%%%%%%%%%%%%%%%%%%%%%%%%%%%%%%%%%%%%%%%%%%%%%

\begin{document}

%%%%%%%%%%%%%%%%%%%%%%%%%%%%%%%%%%%%%%%%%%%%%%%%%%%%%%%%%%%%%%%%%%%%%%%%%%%%%%
%%% Here starts the poster
%%%---------------------------------------------------------------------------
%%% Format it to your taste with the options
%%%%%%%%%%%%%%%%%%%%%%%%%%%%%%%%%%%%%%%%%%%%%%%%%%%%%%%%%%%%%%%%%%%%%%%%%%%%%%
% Define some colors

%\definecolor{lightblue}{cmyk}{0.83,0.24,0,0.12}
\definecolor{lightblue}{rgb}{0.145,0.6666,1}

% Draw a video
\newlength{\FSZ}
\newcommand{\drawvideo}[3]{% [0 0.25 0.5 0.75 1 1.25 1.5]
   \noindent\pgfmathsetlength{\FSZ}{\linewidth/#2}
   \begin{tikzpicture}[outer sep=0pt,inner sep=0pt,x=\FSZ,y=\FSZ]
   \draw[color=lightblue!50!black] (0,0) node[outer sep=0pt,inner sep=0pt,text width=\linewidth,minimum height=0] (video) {\noindent#3};
   \path [fill=lightblue!50!black,line width=0pt] 
     (video.north west) rectangle ([yshift=\FSZ] video.north east) 
    \foreach \x in {1,2,...,#2} {
      {[rounded corners=0.6] ($(video.north west)+(-0.7,0.8)+(\x,0)$) rectangle +(0.4,-0.6)}
    }
;
   \path [fill=lightblue!50!black,line width=0pt] 
     ([yshift=-1\FSZ] video.south west) rectangle (video.south east) 
    \foreach \x in {1,2,...,#2} {
      {[rounded corners=0.6] ($(video.south west)+(-0.7,-0.2)+(\x,0)$) rectangle +(0.4,-0.6)}
    }
;
   \foreach \x in {1,...,#1} {
     \draw[color=lightblue!50!black] ([xshift=\x\linewidth/#1] video.north west) -- ([xshift=\x\linewidth/#1] video.south west);
   }
   \foreach \x in {0,#1} {
     \draw[color=lightblue!50!black] ([xshift=\x\linewidth/#1,yshift=1\FSZ] video.north west) -- ([xshift=\x\linewidth/#1,yshift=-1\FSZ] video.south west);
   }
   \end{tikzpicture}
}

\hyphenation{resolution occlusions}
%%
\begin{poster}%
  % Poster Options
  {
  % Show grid to help with alignment
  grid=false,
  % Column spacing
  colspacing=1em,
  columns=3,
  % Color style
  bgColorOne=gray,
  bgColorTwo=white,
  borderColor=black,
  headerColorOne=white,
  headerColorTwo=lightblue,
  headerFontColor=black,
  boxColorOne=white,
  boxColorTwo=lightblue,
  % Format of textbox
  textborder=roundedleft,
  % Format of text header
  eyecatcher=true,
  headerborder=closed,
  headerheight=0.1\textheight,
 % textfont=\sc
  headershape=roundedright,
  headershade=shadelr,
  headerfont=\Large\textsc, %Sans Serif,
  textfont=\large{\setlength{\parindent}{1.5em}},
  boxshade=plain,
%  background=shade-tb,
  background=plain,
  linewidth=2pt
  }
  % Eye Catcher
  {\includegraphics[height=8em,keepaspectratio=true]{images/logo_uprrp}} 
  % Title
  {\bf\textsc{On a Class of Permutation Polynomials}\vspace{0.1em}}
  % Authors
  {\color{white}{\textsc{Christian A. Rodr\'{\i}guez; Alex D. Santos; Ivelisse Rubio; Francis Castro; \\ Department of Computer Science, \\ University of Puerto Rico, Rio Piedras Campus}}}
  % University logo
  {% The makebox allows the title to flow into the logo, this is a hack because of the L shaped logo.
    \includegraphics[height=9em,keepaspectratio=true]{images/logo_ccom}
  }

%%%%%%%%%%%%%%%%%%%%%%%%%%%%%%%%%%%%%%%%%%%%%%%%%%%%%%%%%%%%%%%%%%%%%%%%%%%%%%
%%% Now define the boxes that make up the poster
%%%---------------------------------------------------------------------------
%%% Each box has a name and can be placed absolutely or relatively.
%%% The only inconvenience is that you can only specify a relative position 
%%% towards an already declared box. So if you have a box attached to the 
%%% bottom, one to the top and a third one which should be in between, you 
%%% have to specify the top and bottom boxes before you specify the middle 
%%% box.
%%%%%%%%%%%%%%%%%%%%%%%%%%%%%%%%%%%%%%%%%%%%%%%%%%%%%%%%%%%%%%%%%%%%%%%%%%%%%%
    %
    % A coloured circle useful as a bullet with an adjustably strong filling
    \newcommand{\colouredcircle}{%
      \tikz{\useasboundingbox (-0.2em,-0.32em) rectangle(0.2em,0.32em); \draw[draw=black,fill=lightblue,line width=0.03em] (0,0) circle(0.18em);}}

%%%%%%%%%%%%%%%%%%%%%%%%%%%%%%%%%%%%%%%%%%%%%%%%%%%%%%%%%%%%%%%%%%%%%%%%%%%%%%
  \headerbox{Abstract}{name=abstract,column=0,row=0}{
%%%%%%%%%%%%%%%%%%%%%%%%%%%%%%%%%%%%%%%%%%%%%%%%%%%%%%%%%%%%%%%%%%%%%%%%%%%%%%
 \vspace{0.3em}  
 Permutation polynomials over finite fields are important in many applications, for example in cryptography. We want to provide families of polynomials that are rich in permutation polynomials. In particular we study polynomials of the form $F_{a,b}(x) = x^{\frac{q-1}{2}} + a x^{\frac{q+d-1}{d}} + b x$, where $a,b \in \mathbb{F}_{q}^{*}$, $q=p^r$, $p$ prime, and $d \mid (q-1)$.
   
   \vspace{0.3em}
 }\label{Abstract}

%%%%%%%%%%%%%%%%%%%%%%%%%%%%%%%%%%%%%%%%%%%%%%%%%%%%%%%%%%%%%%%%%%%%%%%%%%%%%%
\headerbox{Problem}{name=problem,column=1,row=0}{
  %%%%%%%%%%%%%%%%%%%%%%%%%%%%%%%%%%%%%%%%%%%%%%%%%%%%%%%%%%%%%%%%%%%%%%%%%%%%%%
       \vspace{0.3em}
      {\Large Determine conditions in $a, b$ such that polynomials of the form 
      {\LARGE $$F_{a,b}(x) = x^{\frac{q-1}{2}} + a x^{\frac{q+d-1}{d}} + b x,$$}
      \noindent
      where $a,b \in \mathbb{F}_{q}^{*}$, and $d \mid (q-1)$ give permutations of  $\mathbb{F}_{q}$,
      and determine how many pairs exist for each $d$.}

      \vspace{0.3em}
}\label{Problem}

%%%%%%%%%%%%%%%%%%%%%%%%%%%%%%%%%%%%%%%%%%%%%%%%%%%%%%%%%%%%%%%%%%%%%%%%%%%%%%
  \headerbox{Results}{name=contribution,column=1,below=problem,above=bottom}{
%%%%%%%%%%%%%%%%%%%%%%%%%%%%%%%%%%%%%%%%%%%%%%%%%%%%%%%%%%%%%%%%%%%%%%%%%%%%%%
 \vspace{0.3em}
    The following theorem gives information on the amount of polynomials with the same value set.

    \begin{theorem}\label{el_teorema}
        Fix $n \in \mathbb{N}$, $n \leq q$. The number of polynomials of the form $F_{a,b}(x)$ with $\left\vert V_{F_{a,b}} \right\vert = n$ is a multiple of $d$ if $d$ is even, or a multiple of $2d$ if $d$ is odd.
        \end{theorem}

    \begin{corollary}
          The number of permutation polynomials  over $\mathbb{F}_{q}$ of the form $F_{a,b}(x)$ is a multiple of $d$ if $d$ is even, or a multiple of $2d$ if $d$ is odd.
        \end{corollary}

    Given coeffcients $[a,b]$ for which $F_{a,b}(x)$ is a permutation polynomial of $\mathbb{F}_q$, we can construct a list of $d$ or $2d$ coefficients $[a',b']$ such that $F_{a',b'}(x)$ is also permutation polynomial of $\mathbb{F}_q$ as follows:

    \vspace{1em}
    
    \textbf{Construction}: Let $d|(q-1), d$ odd, and $F_{a,b}(x) = x^{\frac{q+1}{2}} + a x^{\frac{q+d-1}{d}} + b x$ be a permutation polynomial of $\mathbb{F}_{q}$, where $a=\alpha^i$, $b=\alpha^j$. Then $F_{a',b'}(x)$ is also a permutation polynomial for $[a',b'] \in \left\{ \alpha^{i+k (d+2) \frac{q-1}{2d}}, \alpha^{j+k \frac{q-1}{2}} \mid k=1,...,2d-1 \right\}.$

    \vspace{1em}

    \begin{example}
      Fix $d = 3$ and $q = 43$. There exists $48$ pairs $[a,b]$ such that $F_{a,b}(x) = x^{22} + a x^{15} + b x$ s a permutation polynomial. In particular, we know that $F_{1,17}(x) = x^{22} + 1 x^{15} + 17 x$  is a permutation polynomial over $\mathbb{F}_{43}$. Using $1=3^{42}=\alpha^{42}, 17= 3^{38}=\alpha^{38}$, we obtain $5$ other pairs $[a', b']$ and new permutation polynomials $F_{a',b'}(x)$ using our construction:

      \vspace{.1em}

      \begin{align*}
        &[7=\alpha^{42+5\cdot 7},26=\alpha^{38+21}], [6=\alpha^{42+2(5\cdot 7)},17=\alpha^{38+2(21)}], \\
        &[42=\alpha^{42+3(5\cdot 7)},26=\alpha^{38+3(21)}], \\
        &[36=\alpha^{42+4(5\cdot 7)},17=\alpha^{38+4(21)}], \\ 
        &[37=\alpha^{42+5(5\cdot 7)},26=\alpha^{38+5(21)}]
      \end{align*}
    \end{example}

   \vspace{0.3em}
  }\label{Results}

%%%%%%%%%%%%%%%%%%%%%%%%%%%%%%%%%%%%%%%%%%%%%%%%%%%%%%%%%%%%%%%%%%%%%%%%%%%%%%
\headerbox{Applications}{name=applications,column=2,row=0}{
  %%%%%%%%%%%%%%%%%%%%%%%%%%%%%%%%%%%%%%%%%%%%%%%%%%%%%%%%%%%%%%%%%%%%%%%%%%%%%%
 \vspace{0.3em}     
 An example of applications of permutation polynomials over finite fields are RSA-type cryptosystems. In some of these systems secret messages are encoded as elements of a field $\mathbb{F}_{q}$ with a sufficiently large $q$. The encryption operator used for these systems is a permutation of the field $\mathbb{F}_{q}$ and needs to be efficiently computable. It is easy to see that expressing this operator in terms of a permutation polynomial is simple and efficient.
      \vspace{0.3em}
}\label{Applications}

%%%%%%%%%%%%%%%%%%%%%%%%%%%%%%%%%%%%%%%%%%%%%%%%%%%%%%%%%%%%%%%%%%%%%%%%%%%%%%
  \headerbox{Preliminaries}{name=preliminaries,column=0,below=abstract}{
%%%%%%%%%%%%%%%%%%%%%%%%%%%%%%%%%%%%%%%%%%%%%%%%%%%%%%%%%%%%%%%%%%%%%%%%%%%%%%
\vspace{0.3em}

\vspace{0.3em}

\begin{definition*}
  A \textbf{permutation} of a set $A$ is an ordering of the elements of $A$. A function $f: A \rightarrow A$ gives a permutation of $A$ if and only if $f$ is one to one and onto.
\end{definition*}

\vspace{0.3em}

\begin{definition*}
  A \textbf{finite field} $\mathbb{F}_{q}$, $q=p^r$, $p$ prime, is a field with $q=p^r$ elements.
\end{definition*}

\vspace{0.3em}

\begin{definition*}
  A \textbf{primitive root} $\alpha \in \mathbb{F}_q$ is a generator for the multiplicative group $\mathbb{F}_{q}^{\times}$
\end{definition*}

\begin{example}
  Consider the finite field $\mathbb{F}_{7}$. We have that: $3^1 = 3, 3^2 = 2, 3^3 = 6, 3^4 = 4, 3^5 = 5, 3^6 = 1$, so $3$ is a primitive root of $\mathbb{F}_{7}$.
\end{example}

\vspace{0.3em}

\begin{definition*}
  Let $f(x)$ be a polynomial defined over a finite field $\mathbb{F}_{q}$. Then the \textbf{value set} of $f$ is defined as $V_{f} = \left\{f(a) \mid a \in \mathbb{F}_{q} \right\}$
\end{definition*}

\vspace{0.3em}

Note that a polynomial $f(x)$ defined over $\mathbb{F}_{q}$ is a permutation polynomial if and only if  $V_{f} = \mathbb{F}_{q}$.


\begin{example}
  Consider the polynomial $f(x) = x+3$ defined over $\mathbb{F}_{7}$. We have that $f(0) = 3, f(1) = 4, f(2) = 5, f(3) = 6, f(4) = 0, f(5) = 1, f(6) = 2$, so $f(x)$ is a permutation polynomial over $\mathbb{F}_{7}$
\end{example}

   \vspace{0.3em}
}\label{Preliminaries}

%%%%%%%%%%%%%%%%%%%%%%%%%%%%%%%%%%%%%%%%%%%%%%%%%%%%%%%%%%%%%%%%%%%%%%%%%%%%%%
  \headerbox{Motivation}{name=motivation,column=0,below=preliminaries, above=bottom}{
%%%%%%%%%%%%%%%%%%%%%%%%%%%%%%%%%%%%%%%%%%%%%%%%%%%%%%%%%%%%%%%%%%%%%%%%%%%%%%
\vspace{0.3em}
  
  Binomials that produce permutations have been studied extensively. The next case to be studied are trinomials. We have found that within the family of polynomials of the form $$F_{a,b}(x) = x^{\frac{q-1}{2}} + a x^{\frac{q+d-1}{d}} + b x,$$ where $d | (q-1)$  there are many permutation polynomials. We want to find conditions in $[a,b]$ that guarantee that $F_{a,b}(X)$ is a permutation polynomial and count how many permutation polynomials exist in each family.

   \vspace{0.3em}
}\label{Motivation}

%%%%%%%%%%%%%%%%%%%%%%%%%%%%%%%%%%%%%%%%%%%%%%%%%%%%%%%%%%%%%%%%%%%%%%%%%%%%%%
  \headerbox{Ongoing Work}{name=future work,column=2,below=applications}{
%%%%%%%%%%%%%%%%%%%%%%%%%%%%%%%%%%%%%%%%%%%%%%%%%%%%%%%%%%%%%%%%%%%%%%%%%%%%%%
\vspace{0.3em}

Knowing a permutation polynomial we can construct $d$ or $2d$ of them (depending on the parity of $d$). We still need to characterize which polynomials are permutation polynomials. For this,  we are studying the size of the value sets of $F_{a,b}(x)$. We divide the value set into subsets:

  \begin{definition*}
      Let $F_{a,b}(x) = x^{\frac{q-1}{2}} + a x^{\frac{q+d-1}{d}} + b x$ be a polynomial defined over $\mathbb{F}_{q}$, where $d \mid (q-1)$. We define the sets $A_l = \left\{F_{a,b}(\alpha^{d k+l}) \mid k=0,...,\frac{q-1}{d}\right\}$ for $l=0,...,d-1$, where $\alpha$ is a primitive root of $\mathbb{F}_{q}$.
    \end{definition*}

    For these subsets we have proved the following lemmas

    \begin{lemma}\label{tamanos_conjuntos}
      Let $F_{a,b}(x)$ be defined over $\mathbb{F}_{q}$ and $A_l$ be defined as above. We have that $\left\vert A_l \right\vert = \frac{q-1}{d}$ or $A_l = \left\{ 0 \right\}$
    \end{lemma}

    \begin{lemma}\label{conjuntos_disjuntos}
      Let $F_{a,b}(x)$ be defined over $\mathbb{F}_{q}$. The sets $A_l$ defined  above are such that, for $l \not = k , \ A_l \cap A_k = \emptyset$ or $A_l = A_k$.
    \end{lemma}

    \begin{proposition}
      Let $F_{a,b}(x)$ be defined over $\mathbb{F}_{q}$ and $A_l$ be defined as above. $F_{a,b}(x)$ is a permutation polynomial if and only if $A_l \neq \{0\}$ and $A_l \cap A_k = \emptyset$ for $ 0 \leq l,k \leq d-1$. 
    \end{proposition}
    \textbf{Aim:}
    \begin{itemize}
      \item Find necessary and sufficient conditions on the coefficients $a=\alpha^i$, $b=\alpha^j$ such that $A_l \neq \{0\}$ and $A_l \cap A_k = \emptyset$
      \item Study our results on the family of polynomials of the form $F_{a,b}(x) = x^{\frac{q+1}{2}+m} + a x^{\frac{q+d-1}{d}+m} + b x^{m}$
    \end{itemize}


   \vspace{0.3em}
  }\label{Future Work}

%%%%%%%%%%%%%%%%%%%%%%%%%%%%%%%%%%%%%%%%%%%%%%%%%%%%%%%%%%%%%%%%%%%%%%%%%%%%%%
  \headerbox{References}{name=applications,column=2,above=bottom,below=future work}{
%%%%%%%%%%%%%%%%%%%%%%%%%%%%%%%%%%%%%%%%%%%%%%%%%%%%%%%%%%%%%%%%%%%%%%%%%%%%%%
  {\small \begin{itemize}
      \item Lidl, Rudolf, and Harald Niederreiter. \textit{Finite Fields}. Reading, Mass.: Addison-Wesley Pub. Co., Advanced Book Program/World Science Division, 1983. Print. 
      \item Wan, D., Lidl, R. \textit{Permutation Polynomials of the Form $x^{r}f(x^{\frac{q-1}{d}})$ and Their Group Structure}. Mh. Math. 112, 149-163 (1991).
      \item Mullen, G., Stevens H. \textit{Polynomial Functions (mod $m$)}. Acta Math. Hung. 44(3-4) (1984), 237-241.
    \end{itemize}}
   \vspace{0.3em}
  }\label{References}

\end{poster}

\end{document}
