\documentclass[landscape,final,paperwidth=48in,paperheight=36in]{baposter}

\usepackage{calc}
\usepackage{graphicx}
\usepackage{amsmath}
\usepackage{amssymb}
\usepackage{relsize}
\usepackage{multirow}
\usepackage{rotating}
\usepackage{bm}
\usepackage{url} 

\usepackage{amsthm} 
\usepackage{amssymb}
\usepackage{amsmath}

\usepackage{graphicx}

\newtheorem{corollary}{Corollary}
\newtheorem{conjecture}{Conjecture}
\newtheorem*{example*}{Example}
\newtheorem*{definition*}{Definition}
\newtheorem{definition}{Definition}
\newtheorem{proposition}{Proposition}
\newtheorem{lemma}{Lemma}
\newtheorem{theorem}{Theorem}

\usepackage{graphicx}
\usepackage{multicol}

%\usepackage{times}
%\usepackage{helvet}
%\usepackage{bookman}
\usepackage{palatino}

\newcommand{\captionfont}{\footnotesize}

\graphicspath{{images/}{../images/}}
\usetikzlibrary{calc}

\newcommand{\SET}[1]  {\ensuremath{\mathcal{#1}}}
\newcommand{\MAT}[1]  {\ensuremath{\boldsymbol{#1}}}
\newcommand{\VEC}[1]  {\ensuremath{\boldsymbol{#1}}}
\newcommand{\Video}{\SET{V}}
\newcommand{\video}{\VEC{f}}
\newcommand{\track}{x}
\newcommand{\Track}{\SET T}
\newcommand{\LMs}{\SET L}
\newcommand{\lm}{l}
\newcommand{\PosE}{\SET P}
\newcommand{\posE}{\VEC p}
\newcommand{\negE}{\VEC n}
\newcommand{\NegE}{\SET N}
\newcommand{\Occluded}{\SET O}
\newcommand{\occluded}{o}

%%%%%%%%%%%%%%%%%%%%%%%%%%%%%%%%%%%%%%%%%%%%%%%%%%%%%%%%%%%%%%%%%%%%%%%%%%%%%%%%
%%%% Some math symbols used in the text
%%%%%%%%%%%%%%%%%%%%%%%%%%%%%%%%%%%%%%%%%%%%%%%%%%%%%%%%%%%%%%%%%%%%%%%%%%%%%%%%

%%%%%%%%%%%%%%%%%%%%%%%%%%%%%%%%%%%%%%%%%%%%%%%%%%%%%%%%%%%%%%%%%%%%%%%%%%%%%%%%
% Multicol Settings
%%%%%%%%%%%%%%%%%%%%%%%%%%%%%%%%%%%%%%%%%%%%%%%%%%%%%%%%%%%%%%%%%%%%%%%%%%%%%%%%
\setlength{\columnsep}{1.5em}
\setlength{\columnseprule}{0mm}

%%%%%%%%%%%%%%%%%%%%%%%%%%%%%%%%%%%%%%%%%%%%%%%%%%%%%%%%%%%%%%%%%%%%%%%%%%%%%%%%
% Save space in lists. Use this after the opening of the list
%%%%%%%%%%%%%%%%%%%%%%%%%%%%%%%%%%%%%%%%%%%%%%%%%%%%%%%%%%%%%%%%%%%%%%%%%%%%%%%%
\newcommand{\compresslist}{%
\setlength{\itemsep}{1pt}%
\setlength{\parskip}{0pt}%
\setlength{\parsep}{0pt}%
}

%%%%%%%%%%%%%%%%%%%%%%%%%%%%%%%%%%%%%%%%%%%%%%%%%%%%%%%%%%%%%%%%%%%%%%%%%%%%%%
%%% Begin of Document
%%%%%%%%%%%%%%%%%%%%%%%%%%%%%%%%%%%%%%%%%%%%%%%%%%%%%%%%%%%%%%%%%%%%%%%%%%%%%%

\begin{document}

%%%%%%%%%%%%%%%%%%%%%%%%%%%%%%%%%%%%%%%%%%%%%%%%%%%%%%%%%%%%%%%%%%%%%%%%%%%%%%
%%% Here starts the poster
%%%---------------------------------------------------------------------------
%%% Format it to your taste with the options
%%%%%%%%%%%%%%%%%%%%%%%%%%%%%%%%%%%%%%%%%%%%%%%%%%%%%%%%%%%%%%%%%%%%%%%%%%%%%%
% Define some colors

%\definecolor{lightblue}{cmyk}{0.83,0.24,0,0.12}
\definecolor{lightblue}{rgb}{0.145,0.6666,1}

% Draw a video
\newlength{\FSZ}
\newcommand{\drawvideo}[3]{% [0 0.25 0.5 0.75 1 1.25 1.5]
   \noindent\pgfmathsetlength{\FSZ}{\linewidth/#2}
   \begin{tikzpicture}[outer sep=0pt,inner sep=0pt,x=\FSZ,y=\FSZ]
   \draw[color=lightblue!50!black] (0,0) node[outer sep=0pt,inner sep=0pt,text width=\linewidth,minimum height=0] (video) {\noindent#3};
   \path [fill=lightblue!50!black,line width=0pt] 
     (video.north west) rectangle ([yshift=\FSZ] video.north east) 
    \foreach \x in {1,2,...,#2} {
      {[rounded corners=0.6] ($(video.north west)+(-0.7,0.8)+(\x,0)$) rectangle +(0.4,-0.6)}
    }
;
   \path [fill=lightblue!50!black,line width=0pt] 
     ([yshift=-1\FSZ] video.south west) rectangle (video.south east) 
    \foreach \x in {1,2,...,#2} {
      {[rounded corners=0.6] ($(video.south west)+(-0.7,-0.2)+(\x,0)$) rectangle +(0.4,-0.6)}
    }
;
   \foreach \x in {1,...,#1} {
     \draw[color=lightblue!50!black] ([xshift=\x\linewidth/#1] video.north west) -- ([xshift=\x\linewidth/#1] video.south west);
   }
   \foreach \x in {0,#1} {
     \draw[color=lightblue!50!black] ([xshift=\x\linewidth/#1,yshift=1\FSZ] video.north west) -- ([xshift=\x\linewidth/#1,yshift=-1\FSZ] video.south west);
   }
   \end{tikzpicture}
}

\hyphenation{resolution occlusions}
%%
\begin{poster}%
  % Poster Options
  {
  % Show grid to help with alignment
  grid=false,
  % Column spacing
  colspacing=1em,
  columns=3,
  % Color style
  bgColorOne=gray,
  bgColorTwo=white,
  borderColor=black,
  headerColorOne=white,
  headerColorTwo=lightblue,
  headerFontColor=black,
  boxColorOne=white,
  boxColorTwo=lightblue,
  % Format of textbox
  textborder=roundedleft,
  % Format of text header
  eyecatcher=true,
  headerborder=closed,
  headerheight=0.1\textheight,
 % textfont=\sc
  headershape=roundedright,
  headershade=shadelr,
  headerfont=\Large\textsc, %Sans Serif,
  textfont=\large{\setlength{\parindent}{1.5em}},
  boxshade=plain,
%  background=shade-tb,
  background=plain,
  linewidth=2pt
  }
  % Eye Catcher
  {\includegraphics[height=8em,keepaspectratio=true]{images/logo_uprrp}} 
  % Title
  {\bf\textsc{On a Class of Permutation Polynomials}\vspace{0.1em}}
  % Authors
  {\color{white}{\textsc{Christian A. Rodr\'{\i}guez; Alex D. Santos; Ivelisse Rubio; Francis Castro; \\ Department of Computer Science, \\ University of Puerto Rico, Rio Piedras Campus}}}
  % University logo
  {% The makebox allows the title to flow into the logo, this is a hack because of the L shaped logo.
    \includegraphics[height=9em,keepaspectratio=true]{images/logo_ccom}
  }

%%%%%%%%%%%%%%%%%%%%%%%%%%%%%%%%%%%%%%%%%%%%%%%%%%%%%%%%%%%%%%%%%%%%%%%%%%%%%%
%%% Now define the boxes that make up the poster
%%%---------------------------------------------------------------------------
%%% Each box has a name and can be placed absolutely or relatively.
%%% The only inconvenience is that you can only specify a relative position 
%%% towards an already declared box. So if you have a box attached to the 
%%% bottom, one to the top and a third one which should be in between, you 
%%% have to specify the top and bottom boxes before you specify the middle 
%%% box.
%%%%%%%%%%%%%%%%%%%%%%%%%%%%%%%%%%%%%%%%%%%%%%%%%%%%%%%%%%%%%%%%%%%%%%%%%%%%%%
    %
    % A coloured circle useful as a bullet with an adjustably strong filling
    \newcommand{\colouredcircle}{%
      \tikz{\useasboundingbox (-0.2em,-0.32em) rectangle(0.2em,0.32em); \draw[draw=black,fill=lightblue,line width=0.03em] (0,0) circle(0.18em);}}

%%%%%%%%%%%%%%%%%%%%%%%%%%%%%%%%%%%%%%%%%%%%%%%%%%%%%%%%%%%%%%%%%%%%%%%%%%%%%%
  \headerbox{Abstract}{name=abstract,column=0,row=0}{
%%%%%%%%%%%%%%%%%%%%%%%%%%%%%%%%%%%%%%%%%%%%%%%%%%%%%%%%%%%%%%%%%%%%%%%%%%%%%%
 \vspace{0.3em}  
 Given $q=p^r$, $d_1$ and $d_2$, we construct partitions of polynomials of the form $F_{a,b}(X) =X\left(X^{\frac{q-1}{d_1}} + a X^{\frac{q-1}{d_2}} + b \right)$, where $a,b \in \mathbb{F}_{q}^{*}$, that have value sets of the same cardinality. As a consequence we provide families of permutation polynomials and of polynomials with small value sets. 
   
   \vspace{0.3em}
 }\label{Abstract}

%%%%%%%%%%%%%%%%%%%%%%%%%%%%%%%%%%%%%%%%%%%%%%%%%%%%%%%%%%%%%%%%%%%%%%%%%%%%%%
\headerbox{Problem}{name=problem,column=1,row=0}{
  %%%%%%%%%%%%%%%%%%%%%%%%%%%%%%%%%%%%%%%%%%%%%%%%%%%%%%%%%%%%%%%%%%%%%%%%%%%%%%
       \vspace{0.3em}
      {\Large Study the value set of polynomials of the form {\LARGE $$F_{a,b}(X) = X(X^{\frac{q-1}{d_1}} + aX^{\frac{q-1}{d_2}} +b)$$} over finite fields $\mathbb{F}_q$ and determine conditions in $a,b$ such that they are permutation polynomials.}

      \vspace{0.3em}
}\label{Problem}

%%%%%%%%%%%%%%%%%%%%%%%%%%%%%%%%%%%%%%%%%%%%%%%%%%%%%%%%%%%%%%%%%%%%%%%%%%%%%%
  \headerbox{Results - Value Sets}{name=contribution,column=1,below=problem,above=bottom}{
%%%%%%%%%%%%%%%%%%%%%%%%%%%%%%%%%%%%%%%%%%%%%%%%%%%%%%%%%%%%%%%%%%%%%%%%%%%%%%

    We define a relation to construct equivalence classes of polynomials with value sets of the same cardinality.

\begin{definition}\label{relacion}

  Let $a = \alpha^i, b = \alpha^j$ and $\sim$ be the relation in $\mathbb{F}_q^* \times \mathbb{F}_q^*$ defined by $(a,b) \sim (a', b')$ 
  $\Longleftrightarrow a' = \alpha^{i+h(\frac{q-1}{d_1} - \frac{q-1}{d_2})}, b' = \alpha^{j+h(\frac{q-1}{d_1})}$, where $h \in \mathbb{Z}$

\end{definition}
  \begin{example*}
    Let $q = 13, d_1 = 2, d_2 = 3$, then we have $\alpha = 2$ and take $a = 2^2 = 4, b = 2^3 = 8$. Now $(a,b) \sim (a',b')$ if and only if
    $a' = \alpha^{2+2h}, b' = \alpha^{3+6h}$. For example $(2^2,2^3) \sim (2^{2+2},2^{3+6})$
  \end{example*}
\begin{lemma}
  
  The relation $\sim$ in Def~\ref{relacion} is an equivalence relation in $\mathbb{F}_q^* \times \mathbb{F}_q^*$.

\end{lemma}


  The equivalence relation defined above induces an equivalence relation in the set of polynomials of the form $F_{a,b}(X) = X(X^{\frac{q-1}{d_1}} + aX^{\frac{q-1}{d_2}} +b)$ with equivalence classes $[F_{a,b}] = [F_{\alpha^i, \alpha^j}] = \left\{\ F_{a',b'} | a' = \alpha^{i+h(\frac{q-1}{d_1} - \frac{q-1}{d_2})}, b' = \alpha^{j+h(\frac{q-1}{d_1})} \right\}$.

  This provides a construction for polynomials with value sets of the same cardinality. 
\begin{theorem}
  
  Suppose that $F_{a,b} \sim F_{a',b'}$ where $\sim$ is the equivalence relation of Lemma 1. Then $|V(F_{a,b})| = |V(F_{a',b'})|$

\end{theorem}

\includegraphics[width=11cm, height=5cm]{clases}

{\small * The polynomials within each cell have value sets of the same size. The size of the value sets associated to different cells might or might not be equal.}

 % \begin{example*}
 %    Let $q = 13, d_1 = 2, d_2 = 3, a = 4, b = 8$. Since $(4,8) \sim (3,5)$ we have that $|V(F_{4, 8})| = |V(F_{3, 5})|$. In fact $V(F_{4, 8}) = \left\{0, 1, 2, 3, 10, 11, 12\right\}$, $V(F_{3, 5}) = \left\{0, 2, 4, 6, 7, 9, 11\right\}$. Note that even though the cardinalities are equal the value sets are not.
 %  \end{example*}
}\label{Results}

%%%%%%%%%%%%%%%%%%%%%%%%%%%%%%%%%%%%%%%%%%%%%%%%%%%%%%%%%%%%%%%%%%%%%%%%%%%%%%
  \headerbox{Preliminaries}{name=preliminaries,column=0,below=abstract}{
%%%%%%%%%%%%%%%%%%%%%%%%%%%%%%%%%%%%%%%%%%%%%%%%%%%%%%%%%%%%%%%%%%%%%%%%%%%%%%
\vspace{0.3em}

\begin{definition*}
  A \textbf{permutation} of a set $A$ is an ordering of the elements of $A$. A function $f: A \rightarrow A$ gives a permutation of $A$ if and only if $f$ is one to one and onto.
\end{definition*}

\begin{definition*}
  A \textbf{finite field} $\mathbb{F}_{q}$, $q=p^r$, $p$ prime, is a field with $q=p^r$ elements.
\end{definition*}

\begin{definition*}
  A \textbf{primitive root} $\alpha \in \mathbb{F}_q$ is a generator for the multiplicative group $\mathbb{F}_{q}^{*}$
\end{definition*}

\begin{example*}
  Consider the finite field $\mathbb{F}_{7}$. We have that: $3^1 = 3, 3^2 = 2, 3^3 = 6, 3^4 = 4, 3^5 = 5, 3^6 = 1$, so $3$ is a primitive root of $\mathbb{F}_{7}$.
\end{example*}

\begin{definition*}
  Let $f(x)$ be a polynomial defined over a finite field $\mathbb{F}_{q}$. Then the \textbf{value set} of $f$ is defined as $V_{f} = \left\{f(a) \mid a \in \mathbb{F}_{q} \right\}$
\end{definition*}

Note that a polynomial $f(x)$ defined over $\mathbb{F}_{q}$ is a permutation polynomial if and only if  $V_{f} = \mathbb{F}_{q}$.

   \vspace{0.3em}
}\label{Preliminaries}

%%%%%%%%%%%%%%%%%%%%%%%%%%%%%%%%%%%%%%%%%%%%%%%%%%%%%%%%%%%%%%%%%%%%%%%%%%%%%%
  \headerbox{Motivation}{name=motivation,column=0,below=preliminaries}{
%%%%%%%%%%%%%%%%%%%%%%%%%%%%%%%%%%%%%%%%%%%%%%%%%%%%%%%%%%%%%%%%%%%%%%%%%%%%%%
  
  Binomials that produce permutations of finite fields have been studied extensively. The next case to be studied are trinomials. We have found that within the family of polynomials of the form $F_{a,b}(X) =X\left(X^{\frac{q-1}{d_1}} + a X^{\frac{q-1}{d_2}} + b \right),$ where $d_1 | (q-1)$ and $d_2 | (q-1)$  there are many permutation polynomials. We want to find conditions in $a,b$ that guarantee that $F_{a,b}(X)$ is a permutation polynomial and count how many permutation polynomials exist in each family.

}\label{Motivation}

%%%%%%%%%%%%%%%%%%%%%%%%%%%%%%%%%%%%%%%%%%%%%%%%%%%%%%%%%%%%%%%%%%%%%%%%%%%%%%
\headerbox{Applications}{name=applications,column=0,below=motivation,above=bottom}{
  %%%%%%%%%%%%%%%%%%%%%%%%%%%%%%%%%%%%%%%%%%%%%%%%%%%%%%%%%%%%%%%%%%%%%%%%%%%%%%
 \begin{itemize}

 \item The encryption operator of some cryptosystems is a permutation of a finite field $\mathbb{F}_{q}$ and needs to be efficiently computable. Expressing this operator in terms of a polynomial is simple and efficient.

 \item Polynomials with minimal value sets have been connected with curves with a large number of rational points.  

 \end{itemize}
}\label{Applications}

%%%%%%%%%%%%%%%%%%%%%%%%%%%%%%%%%%%%%%%%%%%%%%%%%%%%%%%%%%%%%%%%%%%%%%%%%%%%%%
  \headerbox{Results - Permutation Polynomials}{name=work,column=2}{
%%%%%%%%%%%%%%%%%%%%%%%%%%%%%%%%%%%%%%%%%%%%%%%%%%%%%%%%%%%%%%%%%%%%%%%%%%%%%%

The number of polynomials in each cell is the least common multiple of $d_1$ and $d_2$.

$$(4, 3), (34, 34), (30, 3), (33, 34), (3, 3), (7, 34)$$

\begin{proposition}
  
  $|[F_{a, b}]| = lcm(d_1,d_2)$.

\end{proposition}

 % \begin{example*}
 %    Let $q = 13, d_1 = 2, d_2 = 3, a = 4, b = 8$. Note that $lcm(2,3) = 6$. These are the elements of $[F_{a,b}]$:
 %    $$ F_{4, 8}, F_{3, 5}, F_{12, 8}, F_{9, 5}, F_{10, 8}, F_{1, 5}, F_{4,8} $$
 %    All of these polynomials have the same cardinality, although they have different value sets.
 %  \end{example*}

There can be more than one cell with value sets of the same cardinality.
  
    \begin{theorem}\label{laprop}
    The number of polynomials of the form $F_{a, b}(x)$ with $|V_{a, b}| = n$ is a multiple of $lcm(d_1,d_2)$.
  \end{theorem}

A direct result of Theorem~\ref{laprop} is the specific case where $|V_{a, b}| = q$, when we have permutation polynomials. Also, the construction above provides a way to obtain families of permutation polynomials.

  \begin{corollary}
    The number of permutation polynomials of the form $F_{a, b}(x)$ is a multiple of $lcm(d_1,d_2)$.
  \end{corollary}

  \begin{center}
    \begin{tabular}{ | l | c || r | }
      \hline 
      $\mathbb{F}_q$  & $d_1, d_2$ & number of pp \\
      \hline
      $\mathbb{F}_{11}$ &    1, 2    &  36 \\
      $\mathbb{F}_{11}$ &    2, 2    &  46 \\
      $\mathbb{F}_{11}$ &   10, 5    &  10 \\  
      \hline
      $\mathbb{F}_{31}$ &    5, 1    &  30 \\
      $\mathbb{F}_{31}$ &    6, 2    &  18 \\
      $\mathbb{F}_{31}$ &   10, 10   &  30 \\
      \hline                   
      $\mathbb{F}_{37}$ &    2, 3    &  12 \\
      $\mathbb{F}_{37}$ &    2, 4    &  132 \\
      $\mathbb{F}_{37}$ &    6, 6    &  36 \\
      \hline 
    \end{tabular}
  \end{center}

  }\label{Future Work}

%%%%%%%%%%%%%%%%%%%%%%%%%%%%%%%%%%%%%%%%%%%%%%%%%%%%%%%%%%%%%%%%%%%%%%%%%%%%%%
\headerbox{Ongoing Work}{name=future,column=2,below=work}{
  %%%%%%%%%%%%%%%%%%%%%%%%%%%%%%%%%%%%%%%%%%%%%%%%%%%%%%%%%%%%%%%%%%%%%%%%%%%%%%
    \begin{itemize}
    \item Study our results on the family of polynomials of the form $F_{a,b}(X) = X^m(X^{\frac{q-1}{d_1}} + aX^{\frac{q-1}{d_2}} +b)$
    \item Find the possible cardinalities of the value sets.
    \item Find necessary and sufficient conditions such that $V_{a,b} = \mathbb{F}_q$ and $V_{a,b}$ is of minimal cardinality.
  \end{itemize}
}\label{Applications}


%%%%%%%%%%%%%%%%%%%%%%%%%%%%%%%%%%%%%%%%%%%%%%%%%%%%%%%%%%%%%%%%%%%%%%%%%%%%%%
  \headerbox{References}{name=references,column=2,above=bottom,below=future}{
%%%%%%%%%%%%%%%%%%%%%%%%%%%%%%%%%%%%%%%%%%%%%%%%%%%%%%%%%%%%%%%%%%%%%%%%%%%%%%
  \begingroup
  \renewcommand{\section}[2]{}%
  {\small \begin{thebibliography}{}
      \bibitem{main} Panario, D., Mullen, G., \textit{Handbook of Finite Fields}. CRC Press (2013).
      \bibitem{main} Wan, D., Lidl, R. \textit{Permutation Polynomials of the Form $x^{r}f(x^{\frac{q-1}{d}})$ and Their Group Structure}. Mh. Math. 112, 149-163 (1991).
      \bibitem{main} Mullen, G., Stevens H. \textit{Polynomial Functions (mod $m$)}. Acta Math. Hung. 44(3-4) (1984), 237-241.
    \end{thebibliography}}
   \vspace{0.3em}
   \endgroup
  }\label{References}

\end{poster}

\end{document}
